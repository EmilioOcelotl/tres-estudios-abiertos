
% algo tipo metodología
% Aquí tendría que escribir algo de la relación entre la parte escrita de la investigación y la obra 

\section*{Escritura y Ejecución}

Para su ejecución, el proyecto alude a la investigación artística y extiende la idea de \textit{loop} o bucle para hablar de la relación que existe entre investigación y práctica artística cuando el investigador y artista son la misma persona, tal y como describe \cite{ocelotlMas}. Esta forma de trabajo retroalimenta la escritura investigativa con la práctica artística y viceversa. Consideramos que esta perspectiva revela aspectos que el distanciamiento convencional no toma en cuenta. % Creo que por aquí tengo que citar a Lopez Cano

Abordamos procesos artísticos que involucran música, software y computadoras. La investigación no busca realizar mediciones para establecer parámetros de ligereza o eficacia el software resultante. En todo caso, aprovecha la lógica tecnológica de la programación y del procesamiento de información para encontrar soluciones para la investigación y la reflexión.

Rescatamos el uso del repositorio como una estrategia para escribir software, como un respaldo para el trabajo colaborativo y como una documentación que puede modificarse y consultarse en el tiempo. Para la presente investigación, la noción de repositorio de código es central ya que nos permite adentarnos en \textit{las tripas del software} y contemplar al código como un recurso de y para la investigación. En este sentido la escritura y la consulta de software puede realizarse con sistemas distribuidos de control de versiones como \textit{Git}\footnote{``\textit{Git} es un sistema distribuido de control de versiones libre y abierto diseñado para tratar con todo, desde proyectos pequeños hasta proyectos muy grandes con velocidad y eficiencia.'' \url{https://git-scm.com/} (Consultado el \today)}. Con esto, nos adentramos en la discusión sobre el carácter efímero del software y de piezas artísticas para el navegador. 

Aprovechamos la lógica de estos sistemas para construir un entramado que pueda dar cuenta, por un lado, del proceso creativo con los respositorios de las piezas que existen en la web y por el otro, el texto que conforma la investigación y que pretende discutir con la contraparte programada. De esta manera, los procesos quedan lo suficientemente abiertos como para interrelacionarse sin perder delimitación y diferenciación. Buscamos extender la propuesta de un momento anterior de investigación que anuncia la lógica del trabajo con sistemas distribuidos para la investigación artística con tecnología orientada a la creación audivisual.

La ejecución de la investigación consiste en relacionar código, texto y recursos multimedia. En un momento futuro, ponderaremos la implementación de distintos sistemas de escritura de texto/código y elegiremos el que mejor se adapte a los objetivos y alcances del proyecto. Hasta el momento, la investigación considera tres sistemas que de alguna u otra manera se relacionan con la escritura de texto/código: \textit{LaTeX}\footnote{LaTeX es un sistema de composición tipográfica de alta calidad; incluye funcionalidades diseñadas para la producción de documentación técnica y científica. \url{https://www.latex-project.org/} (Consultado el \today)}, \textit{Git} y \textit{JupyterLab}\footnote{\textit{JupyterLab} es un entorno de desarrollo interactivo basado en la web para \textit{notebooks} de \textit{Jupyter}, código y datos. \url{https://jupyter.org/}. (Consultado el \today)}. La elección o combinación de entornos para la escritura de la tesis requerirán una ponderación que tome en cuenta la integración entre texto, código y acoplamiento con el lenguaje de programación principal del proyecto: \textit{Javascript}. De manera complementaria, la investigación contemplará los alcances de diseño, composición tipográfica y estilo personalizados. Como una alternativa al formato de presentación impreso (digital o físico), el proyecto busca que el proceso y el resultado pueda ser compilado y consultado en línea como una página web.\footnote{Una prueba de esta propuesta se puede consultar en: \url{https://emilioocelotl.github.io/tres-estudios-abiertos/}}

La ejecución de la investigación coincide con los planteamientos y la delimitación del proceso de reflexión-creación: estos procesos implican múltiples hilos que corren al mismo tiempo y que podemos enunciar de manera general como: tecnológico, estético y de investigación. El uso de conceptos que atraviesen estos rubros nos permitirá desplazarnos a partir de la retroalimentación que se genera entre tecnología y propuesta artística. El vínculo hacia lo reflexivo puede establecerse en las plataformas para escribir código como entornos de desarrollo integrados (IDEs) y sistemas de control de versiones, en la escritura por sí misma como tecnología y en el uso de conceptos y su vinculación con la metáfora. 


