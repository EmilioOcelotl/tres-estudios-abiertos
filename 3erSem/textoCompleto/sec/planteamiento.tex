\section*{Descripción} % Cambiar nombre 

El presente proyecto se inserta en un ecosistema de funcionalidades, exploraciones estéticas, escritura de software y comunidades que de manera directa o indirecta se vinculan con la programación estética. Si partimos de esta relación de agentes, podemos contemplar dos distinciones que permiten localizar a \textit{Tres Estudios Abiertos} en este ecosistema. 

%% De entre los proyectos similares destacamos aquellos que son de nivel alto para responder a la pregunta: ¿Cuál es la diferencia entre los proyectos mencionados y \textit{Tres Estudios Abiertos}?

% Para aclarar el punto de partida del proyecto, tomaremos en consideración dos perspectivas.

La primera está centrada en proyectos específicos y se centra en el uso e implementación de módulos de código en una comunidad que ejecuta, retroalimenta y enriquece al proyecto. Para este caso, destacamos el caso de Hydra \citep{hydra} o TidalCycles \citep{tidal}. La segunda está orientada a la diversidad no focalizada en un proyecto específico con una comunidad orbitante de usuarios. Se enfoca principalmente a la escritura de interfaces de control, para este rubro mencionamos las posibilidades de ecosistemas que albergan lenguajes de programación como Estuary \citep{estuary} y sema-engine \citep{sema} y librerías para el control de motores de audio como INSTRUMENT \citep{instrument} o canon-generator \citep{canongenerator}

El presente proyecto se delimita a partir de una serie de características que son comunes a los proyectos del ecosistema de usuarios y escritores de software. Éstas influyen y delimitan programas e interfaces y conducen la realización de obras e interpretaciones artísticas a partir de notaciones musicales y computacionales, sintaxis de programación, estilos musicales, ruteos de señales similares a los flujos de voltaje e incluso planteamientos de crítica política y social. 

Referimos a este conjunto de características como decisiones de diseño en sintaxis que controlan motores de audio y video y que tienen consecuencias en la estética del resultado sonoro/visual. \textit{Tres Estudios Abiertos} las transita y se adscribe a esta diversidad en la escritura de y con software.

De manera específica, este proyecto propone bibliotecas para un caso específico de sonoridad digital: síntesis granular audiovisual en el navegador. El proyecto opta por este camino en resistencia a la preponderancia del patrón audiovisual basado en relojes y en pos de la densidad, la textura y el gesto como detonador de eventos sonoros. Aprovecha las posiblidades de procesamiento y conexión en red para generar módulos de software. El proyecto implementará módulos de software en piezas audiovisuales para el navegador, cada una, tendrán módulos adicionales que matizarán el esqueleto granular y que aportarán elementos tecnológicos para la diferenciación entre piezas.

La ruta de navegación de este escrito implica cuatro grupos de exploración. Estos niveles serán abordados de manera no lineal y se describen a continuación:  

\begin{itemize}


\item \textbf{Bajo.-} Como parte de la búsqueda por la ligereza y el bajo consumo de recursos, el proyecto busca explorar las posibilidades de la compilación de código fuente directamente en el navegador. Para este fin, el uso de \textit{Web Assembly} será necesario. En este nivel es posible proponer una librería granular compliada para el navegador. 
\item \textbf{Medio.-} Implementación del esqueleto de módulos con \textit{frameworks} dedicados previamente escritos como Web Audio API\footnote{``La API de Audio Web provee un sistema poderoso y versatil para controlar audio en la Web, permitiendo a los desarrolladores escoger fuentes de audio, agregar efectos al audio, crear visualizaciones de audios, aplicar efectos espaciales (como paneo) y mucho más.'' \url{https://developer.mozilla.org/es/docs/Web/API/Web_Audio_API} (Consultado el \today)}, three.js\footnote{``El proyecto de three.js apunta a la creación de una librería 3D fácil de usar, ligera, multinavegador, multipropósito''. \url{https://threejs.org/} (Consultado el \today)} y Icecast\footnote{``\textit{Icecast} es un servidor para transmitir audio y video, actualmente soporta Ogg (Vorbis y Theora), Opus, WebM y MP3. Puede ser usado para crear una estación de radio por Internet o para correr de manera privada una rocola y muchas otras cosas.'' \url{https://icecast.org/} (Consultado el \today)}. Este nivel es de utilidad para bosquejar el posible comportamiento del esqueleto granular con marcos robustos y estables. 
\item \textbf{Alto.-} Implementación de funciones de control que puedan interactuar con el esqueleto de módulos para la granulación de audio y video. Idealmente éstas conformarán la sintaxis de una interfaz de texto que será deducida de \textit{Javascript}. La escritura de software y la referencia a otros proyectos de nivel alto permite tener una idea de cómo se podría resolver el control de librerías y de motores de audio y video. Este nivel permite explorar las posibilidades poéticas del texto como interfaz. Este aspecto también considera la experiencia del usuario / espectador. 
\item \textbf{Ecosistemas.-} El proyecto podría integrarse a la agendas de contribución comunitaria de proyectos como sema-engine \citep{2019_40} y Estuary. Esto permitiría el trabajo modular de manera colectiva en colaboración con otros proyectos similares y nutrirse de la experiencia de diseño de lenguajes para el control de señales de audio e imagen a partir de interfaces de texto. 
  
\end{itemize}

% Si bien el proyecto puede diferenciarse en estos tres niveles, el objetivo al que apunta es la interacción del software escrito en cada fase.

Para la realización de una librería de síntesis granular audiovisual, el proyecto parte de conceptos propuestos por \cite{microsound} y en específico aborda la noción de escala de tiempo para realizar acercamientos o alejamientos que puedan enfocar obra artística, tecnología y que incluso alcancen a plantear un marco de observación social. Esta propuesta se desplaza en un continuo que nos permite observar el sonido en una dimensión microscópica, lo político y lo social en una dimensión macroscópica y las piezas artísticas como un punto intermedio entre estas dos escalas. 

Los casos de estudio estarán alojados en la web. En este sentido, el proyecto pretende resolver el \textit{backend} del proyecto y en específico, busca explorar las posiblidades técnicas y conceptuales de la distribución el web a partir del concepto par-a-par\footnote{\textit{P2P} (par a par) ``La arquitectura de una red distribuida puede ser llamada Par a Par (P-to-P, P2P, ...)   si los participantes comparten una parte de los recursos de su propio software (poder de procesamiento, capacidad de almacenamiento, capacidad de conexión a la red, impresoras,...) Estos recursos compartidos son necesarios para proveer el Servicio y el contenido ofrecido por la red... Estos son accedidos por otros pares directamente sin pasar por entidades intermediarias." \citep{p2p}}. Hemos optado por arquitecturas distribuidas para la realización de las piezas para reducir la carga en la comunicación enetre nodos, para no depender de la centralidad de un servidor y particularmente para compartir flujos de audio y video a través de la web sin restricciones de contenido o forma. El proyecto en este sentido apunta hacia la realización de un sintetizador distribuído alojado en la web y del uso de la noción de nodo y red para la resolución tecnológica de las piezas pero también como una forma de observar la organización social que se relaciona con tecnologías que posibilitan la interacción en espacios digitales. 
