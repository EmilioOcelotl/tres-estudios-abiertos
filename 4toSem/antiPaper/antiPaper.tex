\documentclass[12pt,letterpaper, twocolumn]{article}
\usepackage[utf8]{inputenc}
\usepackage[spanish]{babel}
\usepackage{amsmath}
\usepackage{amsfonts}
\usepackage{amssymb}
\usepackage{graphicx}
\author{Emilio Ocelotl Reyes}
\title{Anti\\
  La ofuscación como motivo, la escritura como rodeo}

\begin{document}

\maketitle

\section*{Resumen}

La presente investigación da cuenta de un espacio de reflexión que parte de un bucle de investigación-producción artística. El resultado práctico de la investigación es un prototipo audiovisual que puede accederse por medio de un navegador web.
Presenta una discusión sobre el uso de motores de audio y gráfico, así como la implementación de eventos y el tiempo en el navegador.
El proyecto gira en torno a un concepto específico: la ofuscación. De manera complementaria busca establecer los desbordamientos resultantes que conducieron a la escritura y a la reflexión. 

\section*{Introducción}

La presente investigación parte de la siguiente pregunta: ¿Puede la ofuscación ser un motivo para la producción artística-tecnológica y un concepto para reflexionarla? 


% \section*{Delimitación y Contexto}

%\section*{Audio e imagen}

%\section*{Eventos}

%\section*{El tiempo en el navegador}

%\section*{La ofuscación como motivo}

%\section*{La escritura como rodeo}

%\section*{Conclusiones}

\end{document}
