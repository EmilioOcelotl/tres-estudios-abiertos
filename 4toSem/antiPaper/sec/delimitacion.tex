
Inicialmente la aplicación fue concebida para ser ejecutada localmente. Dadas las circunstancias específicas de la pandemia de COVID-19, el proyecto migró a una aplicación web.

Una de las consecuencias no buscadas del desarrollo para la web fue la compatibilidad entre sistemas operativos y la cero instalación de entornos y librerías; la aplicación puede ejecutarse con un navegador web actual 

La pieza está alojada en la web y utiliza tone.js como motor de audio y three.js para el despliegue de gráficos tridimensionales. Adicionalmente utiliza: 1) Tensorflow.js para la lectura de puntos de referencia faciales (face-landmark-detection) y 2) algunos módulos adicionales de la librería JSM para el rendereo de efectos de post-proceso de imagen.
