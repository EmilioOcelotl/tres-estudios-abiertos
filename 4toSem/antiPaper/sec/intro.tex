
La ofuscación puede definirse como el acto deliberado de encubrir el significado de una comunicación. Para el caso de la programación y apuntando ideas hacia los estudios del software, la presente investigación toma la noción de ofuscación de aquello que podríamos definir como la \emph{estética del código}\citep{EWD:EWD35} y aquellos programas que exploran ``otros principios estéticos''\citep{obfuscatedCode} además de los convencionales. 

% Para la versión expandida podría citar las referencias 

%% En este punto puedo retomar las ideas de djisktra y de knuth 

Esta definición es el punto de partida de \emph{Anti}, una pieza audiovisual para el navegador que tiene dos objetivos: visibilizar la discusión en torno a el uso de datos y la responsabilidad tecnológica del usuario y 2) actuar como un dispositivo de ofuscación facial y vocal que pueda utilizarse en situaciones de uso cotidiano. 

El maquillaje y el uso de accesorios anti-vigilancia son estrategias analógicas para evitar la detección de rostros. En una situación de protección fuera del entorno digital, incluso una máscara de leds puede cumplir esta función.

El presente proyecto se enfoca los mecanismos de anti-vigilancia que pueden realizarse de manera digital, teniendo a la computadora como un agente intermedio entre dos puntos que desean mantener algún tipo de comunicación gestual y vocal sin que estos puedan detectarse o asociarse a sujetos específicos, sin que esto implique que la comunicación sea completamente ofuscada para los usuarios.

\subsection{Esquema general de la aplicación}

La aplicación cuenta con tres momentos principales: 1) Detección de puntos de referencia faciales 2) motores gráfico y sonoro, 3) materiales y organización y 4) Redirección de flujos de audio y sonido.

% En la versión extendida para la tesis, aquí va el apartado de puesta en marcha y montaje. Considero que para el artículo puede ser innecesariamente extenso

