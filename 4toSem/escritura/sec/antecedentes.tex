
\chapter{Antecedentes}

% este apartado podría tener un orden: de lo más lejano a lo más cercano

% Aquí pienso colocar los antecedentes del proyecto.
% Me pregunto sobre la pertinencia de escribir una especie de estado del arte.
% Había pensado hablar de algunos proyectos cercanos que influyen en el proyecto
% Esto aparece ligeramente enunciado en el artículo de Panorama y en el texto que preparé para el coloquio
% Podría hacer un seguimiento al trabajo que hacen personas cercanas
% Incluso aquí podrían ir los primeros ejercicios en la web 
% Me imagino que aquí podrían aparecer algunos registros anectóticos 

Los antecedentes de esta investigación aluden a una trayectoria que va de la transición de la escritura de software para la realización de sistemas interactivos a la escritura de módulos de software audiovisuales. Estas experiencias toman como premisas la optimización y la ligereza de hardware (por ejemplo, con el uso de computadoras de placa reducida como Raspberry Pi o Jetson Nano ) y la elaboración sistemas ligeros, accesibles y portables para la síntesis y renderización audiovisual en el navegador.

\section{Casos históricos seleccionados}

\subsection{Documenta X}

Conexiones y accesos en muestras que implican espacios físicos y digitales. 

\subsection{Generación unitaria}

la línea de investigación que se remonta a MUSIC N y que desemboca en las ideas que se retoman en algunos entornos de programación creativa con sonido como SuperCollider y Tone.js

\subsection{Jacktrip y la música conectada}

raspis conectadas y jacktrip > el trabajo realizado por CCRMA y en general 

La labor del colectivo Radiador

La radio y la transmisión de sonido con Icecast 

Sonobus y la resolución de problemas de streaming en tiempos de pandemia 

\subsection{Fluxus y openGL}

Para el caso de la imagen, retomo la influencia que tiene en la comunidad de live coding y en mi experienciación del performance audiovisual con la computadora el papel que tuvo el desarrollo Fluxus\footnote{\url{https://gitlab.com/nebogeo/fluxus/}} de Dave Griffiths que se remonta al 2007. Detrás de Fluxus también cabe destacar la importancia un tanto oculta de sistemas de renderización de gráficos por computadora como OpenGL, que actualmente, son el punto de partida de software de alto nivel involucrado con este proyecto como OpenFrameworks y la variante para el navegador, webGL, que implementa la librería Three.js 

\section{Perspectivas}

\textit{Tres Estudios Abiertos} retoma esta incorporación, parte del giro de los nuevos medios y de los estudios del software \citep{manovichlanguage}. Como una extensión del punto de partida, la investigación se adscribe a la escritura con y sobre software \citep{aestheticProgramming}. Atiende al papel que juega la experiencia subjetiva y las implicaciones políticas y sociales en la programación estética \citep{speakingCode}. 


\section{Proyectos Colindantes}

De manera directa, los siguientes proyectos están relacionados con la investigación y se convierten en casos de estudio, en algunas otras ocasiones fungen como antecedentes directos:

\begin{itemize}

\item Caso de estudio. THREE.studies \citep{threestudies} y en específico, las iteraciones threecln \citep{threecln} y threeBEASTs \citep{threeBEASTs}
\item Caso de estudio. Anti \citep{anti} 
\item Antecedente directo. Diálogo IV @ Coloquio Salvador Contreras. Artes Sonoras y Creación Musical en México: Siglo XXI \citep{dialogo}
\item Antecedente directo. Panorama. Escritura de espacios libres e inmersivos para el performance audiovisual \citep{panoramaArticulo}\footnote{Versión en construcción y en línea: \url{https://piranhalab.github.io/panorama/}}
  
\end{itemize}

Algunos otros proyectos cercanos tecnológica, conceptual y estéticamente a \textit{Tres Estudios Abiertos} son: 

\begin{itemize}

\item \textbf{Nivel Bajo:} Ruffbox \citep{ruffbox}, WebAssembly/Rust Tutorial\footnote{\url{https://www.toptal.com/webassembly/webassembly-rust-tutorial-web-audio} (Consultado el \today)} y Flocking \citep{flocking}. 
\item \textbf{Nivel Medio:} \citep{supercolliderweb}, Web Audio API\footnote{\url{https://developer.mozilla.org/es/docs/Web/API/Web_Audio_API}}, Tone.js\footnote{\url{https://tonejs.github.io/}} y supercolliderjs\footnote{\url{https://github.com/crucialfelix/supercolliderjs/}}.
\item \textbf{Nivel Alto:}, Troop \citep{Troop}, flok \citep{flok}, tilt \citep{tilt}, LiveLab \citep{livelab}, Hydra \citep{hydra}, timeNot \citep{timenot} y seis8s \citep{seis8s}
\item \textbf{Ecosistemas:} Estuary \citep{estuary} y sema-engine \citep{sema}.
  
\end{itemize}

\section{Piezas y obras lejanas}

\subsection{The Stage is (A)Live}

The stage is (a)Live\footnote{\url{https://www.geometries.xyz/theStageIsAlive/}} - Johana Chicau y Renick Bell

\subsection{Notas de Ausencia}

Marianne Teixido - Notas de Ausencia

Performances con Hydra

\section{Trilogía de Investigación}

\textit{Tres Estudios Abiertos} forma parte de una trilogía de investigación.

\subsection{Objeto, Paisaje y Efecto}

La primera parte fue Objeto, Paisaje y Efecto \citep{ocelotlLic}, un proyecto de investigación que abordó las nociones de objeto sonoro \citep{schaeffer}, paisaje sonoro\citep{schafer1} y efecto sonoro \citep{augoyard} para considerar a la escucha como un recurso para la investigación sociológica en música y para la investigación social desde el sonido.

\subsection{Cuidado con la Brecha}

Un segundo punto de investigación involucró un proceso de investigación-producción artística \citep{ocelotlMas}. La realización de este proyecto fue un prototipo tecnológico y partió de objetivos que inicialmente estaban propuestos como secundarios pero que más tarde se revelaron como parte del núcleo en la investigación. Estos aspectos son: 1) el proceso de trabajo colaborativo y su implementación con herramientas como git, 2) la reflexión sobre la interacción entre audio e imagen en la composición musical electroacústica y 3) el uso de herramientas libres, personalizadas para la realización de prototipos audiovisuales y para el planteamiento de una observación crítica de procesos creativos donde investigador y artista son el mismo agente. La propuesta de los estudios del software fue incorporada en este momento de investigación. 

\section{Live Coding}

% Antecedentes más extensos

\subsection{Primeros Pasos}

% PhD Thesis: Artist-Programmers and Programming Languages for the Arts - Alex McLean 
Dentro de los antecedentes está la experiencia performática de escribir código al vuelo con fines creativos, audiovisuales y experimentales,

\subsection{Latinoamérica}

Como lo describen \cite{villasenor} para los casos de Barcelona y Ciudad de México.\footnote{Un ejemplo reciente se encuentra en: \url{https://youtu.be/n5kwi4eRAE4}} 

\subsection{Ciudad de México}

\section{Piezas y obras cercanas} % Pendiente, igual y no hay tantas cosas 

THREE.studies

Geometrías híbridas 

\section{PiranhaLab}

Otro antecedente de este proyecto es la práctica y reflexión planteada en colectivo por\textit{PiranhaLab}\footnote{``PiranhaLab es un laboratorio interdisciplinario que trabaja en las tripas del software''. \url{https://piranhalab.github.io/} (Consultado el \today)}.

\subsection{Ciclo de Talleres}

El ciclo de talleres realizado en el Centro de Cultura Digital (CCD) en coparticipación con el Laboratorio de Tecnologías Libres\footnote{Actualmente Laboratorio de Tecnologías Compartidas} permitió plantear dos conclusiones que se heredan a \textit{Tres Estudios Abiertos}: La difuminación de la distinción usuario/desarrollador como una motivación para la escritura de software y la procuración de diversidad en la escritura de software en América Latina.

\subsection{EDGES 2020}

La escritura de espacios para el ciclo de conciertos EDGES 2020 realizado por el Taller de Imágenes en Movimiento del Centro Multimedia (CMM) permitió la exploración de entornos tridimensionales inmersivos en el navegador en el contexto del encierro causado por la pandemia de COVID-19. Técnica y conceptualmente la escritura de estos espacios digitales influye en el presente proyecto. El artículo \textit{Panorama} \citep{panoramaArticulo} hace referencia de manera extensa al ecosistema de espacios y propuestas que también inciden en \textit{Tres Estudios Abiertos}.

% Esto se articula con las actividades colectivas de la sección casos 
