\chapter{Caso 2: Anti}

% Este apartado puede convertirse en un artículo por sí mismo. Describirlo de una manera distinta al tipo contexto, planteamiento, proceso y resultados  

% \href{https://anti.ocelotl.cc}{\includegraphics[width=\textwidth]{img/anti01.png}} % Una imagen que también es hipervínculo 

\begin{figure}[tb]
\centering 
\includegraphics[width=\columnwidth]{../../img/anti01.png} 
\caption[Captura de Anti]{Captura de anti y de la interfaz gráfica que utiliza de \url{https://anti.ocelotl.cc}.} % The text in the square bracket is the caption for the list of figures while the text in the curly brackets is the figure caption
\label{fig:gallery} 
\end{figure}

% Anti es una pieza que parte de la ofuscación para plantear una reflexión sobre la relación que existe entre usuarios y tecnología. Utiliza algunos módulos de \gls{aprendizajeautomatico} (machine learning) 

%\subsection{Contexto}


\section*{Introducción}

\textit{Tres Estudios Abiertos} es un proyecto de investigación que busca adentrarse en nuevas prácticas experimentales y audiovisuales para el navegador. El proyecto busca demostrar que las lógicas de los lenguajes de programación posibilitan formas de pensamiento específicos con consecuencias estéticas que pueden extenderse y aportar a la investigación artística que involucra tecnología. Parte de la programación de un esqueleto de granulación audiovisual complementado con módulos personalizados de software que se implementarán en estudios para el navegador. 

El proyecto se centra en piezas artísticas para el navegador con características específicas: sin instalaciones, sin referencias a dependencias de terceros, basadas en la web, distribuidas y optimizadas para el bajo consumo de recursos de una computadora o una red de computadoras. Considera a \textit{Javascript}\footnote{¿Todavía es posible usar otros lenguajes de programación?} como el lenguaje de programación y busca reflexionar en el nivel bajo, medio y alto de programación. Como parte de la exploración de alto nivel, busca reflexionar sobre las ideas estéticas de los lenguajes de programación para la escritura de interfaces textuales de control. 

Consideramos que hay tres hilos que corren paralelamente en el bucle de ivnestigación-creación: estético, tecnológico e investigativo. En este último encontramos la relación de la escritura del texto resultante con la escritura de software. El vínculo se establece con el uso de sistemas distribuidos de control de versiones para la escritura de software y de texto. El documento resultante busca desbordarse del papel y de la palabra escrita por medio de una versión web. 


\section{Delimitación y contexto}


Inicialmente la aplicación fue concebida para ser ejecutada localmente. Dadas las circunstancias específicas de la pandemia de COVID-19, el proyecto migró a una aplicación web.

Una de las consecuencias no buscadas del desarrollo para la web fue la compatibilidad entre sistemas operativos y la cero instalación de entornos y librerías; la aplicación puede ejecutarse con un navegador web actual 

La pieza está alojada en la web y utiliza tone.js como motor de audio y three.js para el despliegue de gráficos tridimensionales. Adicionalmente utiliza: 1) Tensorflow.js para la lectura de puntos de referencia faciales (face-landmark-detection) y 2) algunos módulos adicionales de la librería JSM para el rendereo de efectos de post-proceso de imagen.


\subsection{Puesta en marcha y montaje}

Anti es una obra que aprovecha las posibilidades de la interactividad en la producción artística con nuevos medios. En este sentido, la relación de la pieza con el especttador coincide con el planteamiento de Hugo Solís que define a esta relación como:

\begin{quote}

``arte que requiere de un input directo por parte de los espectadores para poder considerarse una obra terminada y funcional. Hablamos de obras no lineales en donde al menos uno de los elementos resultantes, comúnmente observable, dependerá de la información que proviene del espectador, directa o indirectamente, ya sea presencial, o remotamente en el momento de la observación, o anticipadamente.''\citep[p.~37]{hugoSolis}

\end{quote}
  
Anti tuvo una modalidad de exhibición presencial que tuvo lugar en el Antiguo Colegio de San Ildefonso.

\begin{figure}[tb]
\centering 
\includegraphics[width=0.7\columnwidth]{../../img/antiExWhite.png} 
\caption[Diagrama de Montaje Anti]{Diagrama de montaje de Anti.} % The text in the square bracket is the caption for the list of figures while the text in the curly brackets is the figure caption
\label{fig:gallery} 
\end{figure}

\subsection{Ecosistema}

Documentación de las piezas que compartieron tiempo y espacio con anti. Preocupaciones compartidas por las consecuencias sociales de la tecnología y el uso responsable de éstas para transformarla, de acuerdo a lo que se mencionó anteriormente de Soon y Cox. Contexto de la desapariciones de los feminicidios en México, la contaminación provocada por el consumo de fast fashion y la relación / interpretación de la realidad a partir de narrativas que contemplan la agencia no-humana. 

La centralidad del individuo 

La idea de múltiples programas de los objetos tecnológicos \citep{latour}

%\section{Descripción técnica general}

Precisión 

\section{Audio e Imagen} % Relevante o puede ir en otro apartado? 

Decisiones de diseño.

\subsection{Audio}

¿Por qué Tone.js? 

\subsection{Imagen}

¿Por qué Three.js?
keypoints y la construcción de meshes > triangulaciones y las convenciones del modelado tridimensional 

Hydra como un módulo para generar texturas en sólidos 

%\subsection{Redirección de flujos de audio y sonido}

\section{Eventos}

\subsection{Puntos y desarrollo}

El desplazamiento del evento musical y la diversificación de los materiales. Peculiaridades del texto, sonido e imagen. Eventos que establecen puntos de salida y de llegada. Preguntas sobre lo que existe entre puntos. La rampa de tiempo como una forma de relacionar eventualidades y desarrollarlas en el tiempo.

\subsection{Transducción de magnitudes}

Relaciones y diferencias. Transducción de magnitudes 

La biblioteca MediaPipe Facemesh devuelve 468 puntos de referencia faciales. ¿Cómo estos puntos son relevantes? 

Promedios de movimiento asociados a todo y a regiones del rostro. 

\subsection{Materiales}

%\subsection{Materiales y organización}

En esta parte se distribuyeron los materiales sonoros, visuales y textuales en eventos que pudieran detonarse a partir de un esquema temporal y espacial a manera de partitura. 

Texturas

\section{El tiempo en el navegador}

\subsection{Tiempo y secuenciación}

Este apartado puede hablar de las diferencias que existen entre las distintas formas de transformar eventos en el tiempo: setInterval, requestAnimationFrame, Tweenjs y los objetos de Tone.js que permiten detonar eventos como si fueran secuenciadores. Hasta el momento se han detectado tres tipos de aproximaciones: 1) Aquella que está medida en microsegundos y que no necesariamente es precisa, 2) aquella que tiene que ver con temporalidades medidas en segundos ( de hecho el diseño de la estructura general de la pieza tomó esta división con punto de partida y 3) la aproximación que coincide con la convención musical basada en golpes por segundo (BPM). 

\subsection{Panorámica}

Partitura

\section{La ofuscación como motivo}

\subsection{Predicciones}

Anti utiliza la biblioteca MediaPipe Facemesh\footnote{MediaPipe Facemesh un un paquete ligero que predice 486 puntos faciales tridimensionales para inferir la superficie geométrica aproximada de una cara humana. Consultado el \today en: Nota: esperar a que suban el paquete} para la detección de puntos de referencia faciales. Estos puntos están optimizados para que las zonas de la cara con mayor gestualidad tengan una densidad de puntos mayor. \citep{kartynnik2019realtime}.

\subsection{Ofuscación audiovisual}

Aquí podría hablar de los aspectos tecnológicos que motivaron la realización de la pieza. Tecnologías diversas que pueden coincidir (no necesariamente lo hacen, por lo menosen términos de la declaratoria y motivación de los proyectos) con la ofuscación facial. Tensorflow y face-landmark-detection. 
te

\subsection{Mediciones}

Plantear la posibilidad de medir el éxito del proyecto con algoritmos de reconocimiento facial. Dar la vuelta a la tecnología como ofuscación y como posibilidad de éxito frente a esa ofuscación. Mediación del diseño humano, decisiones subjetivas basadas en la experiencia. 

\section{La escritura como rodeo} % Otra palabra que no sea rodeo 

Resultados y giros. 
Distinción entre investigación / práctica artística y manifiesto 
Escritura de código.

% Texto académico vs && || manifiesto 

Desplazamiento del sonido 

%\subsection{Planteamiento Inicial}
%\subsection{Proceso y realización}
%\subsection{Resultados}
