\chapter{Introducción}

Motivación, algo breve.

\section{Sobre este documento}

Descripción de la estructura y cómo esto funciona en el papel y en la web 

\section{Contexto de escritura}

Aquí puedo escribir sobre aspectos ``subjetivos'' como la pandemia. Valdría la pena contrarrestar algunas posturas críticas sobre la pandemia y cómo ha esto ha generado consecuencias potenciadores y restrictivas. 

\section{Obras}

Presentación general de las piezas. Hilos conductores comunes y giros que han tomado a partir de la situación. En la realización de las piezas hay una búsqueda técnica en distintos planos. 

\section{Versiones y repositorios}

El uso de git como una herramienta para la escritura y el control de versiones, no solamente de código sino también de la parte escrita de la investigación. La investigación abierta. 

\section{Investigación e inmersión}

Investigación que retoma algunas ideas de las ciencias sociales y de la observación participante. Como desplazar el centro de la investigación al sonido, la imagen o su integración como objetos de conocimiento por sí mismos.

La problematización de la metodología de la investigación como algo presente en el proceso de escritura, como un objetivo secundario que pone en contradicción al investigador inmerso en una actividad práctica. El rodeo como un camino necesario para encontrar rutas no convencionales. 

