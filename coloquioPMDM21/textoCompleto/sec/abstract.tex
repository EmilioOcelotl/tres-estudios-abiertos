
\textit{Tres Estudios Abiertos} es una investigación que aborda prácticas experimentales y audiovisuales para el navegador. Estudia la influencia de los lenguajes de programación en la práctica artística y el aporte que pueden realizar a la investigación artística con tecnología. Como parte del objetivo tecnológico, implementa un esqueleto de granulación audiovisual y módulos personalizados de software que conformarán una colección de estudios para el navegador. 

%Atiende a niveles altos de programación para el control de software por medio de una interfaz de texto. 

Estas piezas tendrán audio y video, estarán alojadas en la web, tendrán una lógica distribuida, serán optimizadas para el bajo consumo de recursos de la computadora y no requerirán instalación ni dependencias adicionales. El proyecto considera a \textit{Javascript} como el lenguaje de programación que permeará al proyecto y que permitirá reflexionar en el nivel bajo, medio y alto de programación. % Posibilidad de enunciar otros lenguajes de programación ? 

El núcleo de la investigación será el bucle que retroalimenta práctica artística y reflexión, es decir, la relación que existirá entre las piezas realizadas, el entramado de software que posibilitará su realización en la web y las implicaciones tecno-sociales de la tecnología principal de este proyecto: los lenguajes de programación. 
