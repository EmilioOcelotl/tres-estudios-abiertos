\section*{Avances y Práctica}

El contexto de la pandemia de COVID-19 ha influído directamente en la investigación. La realización de piezas para el navegador que puedan responder a premisas bajo la restricción del distanciamiento social ha sido un reto para el momento de arranque de \textit{Tres Estudios Abiertos}. 

Actualmente el proyecto está compuesto de algunos módulos escritos. Algunas instancias se han desprendido los módulos y las primeras aproximaciones a la realización de obras audiovisuales para el navegador.

THREE.studies es una pieza realizada en el marco del programa Resilencia Sonora de Música UNAM. La interpretación (cello eléctrico) en todos los casos estuvo a cargo de Iracema de Andrade. La pieza contó con el apoyo estético y logístico de PiranhaLab (Marianne Teixido y Dorian Sotomayor)- 

Al momento de escritura del presente proyecto la primera instancia del estudio no ha sido estrenada.

Una versión fija ha sido exprandida y presentada en el marco de  BEAST FEaST 2021\footnote{\url{http://www.beast.bham.ac.uk/beast-feast-2021/online-works/}}.

Tras bambalinas el proyecto implicó un proceso de producción y montaje a distancia. Descripción. 

4NT1 y algunos eventos que abonan desde el contexto de la pieza. 

Además de las instancias prácticas del proyecto, la investigación se ha retroalimentado a partir de dos proyectos coexistentes: el Seminario Permamente de Tecnología Músical de la FaM y el texto que el colectivo \textit{PiranhaLab} escribe para la realización de este espacio.

