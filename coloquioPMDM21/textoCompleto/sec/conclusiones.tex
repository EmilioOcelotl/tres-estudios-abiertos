\section*{Conclusiones}

Las restricciones de copresencia del contexto pandémico ha posibilitado exploraciones superficiales que fungen como una observación de la panorámica de los rumbos de la investigación. Consideramos que la investigación tiene momentos abiertos que se resuelven en la realización creativa de piezas. La parte más robusta de la investigación se concentra en los antecedentes y en la detección de proyectos cercanos que delimitan al proyecto en términos de los alcances y retos, tomando en consideración un ecosistema relativamente activo de escritura y uso de software en la web potenciado en el contexto de la pandemia. 

Los avances actuales apuntan a la realización de piezas que hagan uso de sintetizadores granulares audiovisuales en la nube. Esto permite desplazar la carga de procesamiento de audio e imagen a un servidor dedicado o a un conjunto de computadoras conectadas entre sí. Este aspecto permitiría realizar análisis del flujo de audio e imagen detallados sin que esto represente un problema para los recursos de las máquinas que acceden a las piezas y los espacios. Esta solución también podría implicar una alternativa para aprovechar las posibilidades de motores gráficos, de audio y de análisis de información mucho más robustos que aquellos que actualmente están disponibles para el navegador y que funcionan únicamente de forma local. La centralidad del procesamiento de señales y la omisión de la experiencia de usuario ha sido una crítica señalada. 

%El lenguaje de programación seleccionado para el proyecto de investigación está optimizado para la web, una de las posibles implicaciones del trayecto de investigación podría reforzar la presencia de las tecnologías y los planteamientos políticos de las redes distribuidas. 

La presente investigación ha tenido en cuenta la ponderación entre práctica e investigación artística. El lado tecnológico de la investigación, aquel que implica la escritura de software, podría orientarse al bajo nivel de programación. Esta acción permitiría resolver aspectos que \textit{frameworks} dedicados resuelven parcialmente. Sobre la realización de la investigación, destacamos la importancia del bucle y la metafora como estrategias para la observación y la práctica transversal que se retroalimenta mientras se ejecuta. 

% Agenda pendiente.

El epicentro de la escritura podría recaer en el mismo texto/código y no en plataformas de repositorios que pertenecen a empresas que pueden cambiar de giro en cualquier momento. El respaldo del trabajo por medios digitales podría encontrar soluciones actuales en la encriptación. Ambas posibilidades pueden extender la escritura a formas experimentales e integradas al código. 

Como parte de las consecuencias señaladas y no buscadas del proyecto se encuentran el activismo ecológico y los planteamientos éticos del software. Estos dos aspectos no son necesariamente centrales pero pueden tenerse en cuenta dentro de una agenda de acción que se desborda de la práctica artística y la investigación.  

