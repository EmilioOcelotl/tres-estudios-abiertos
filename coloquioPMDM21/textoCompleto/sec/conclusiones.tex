\section*{Conclusiones}

El proyecto se ha visto influído por el contexto pandemico que hasta el momento de escritura, sigue vigente. Una de las preguntas que han rondado la investigación es: ¿Qué pasará con el proyecto en un contexto post-pandemia? 

La restricción del contexto ha posibilitado exploraciones superficiales que fungen como una observación de la panorámica de los posibles rumbos de la investigación, sobre todo en lo que implica la escritura de software. 

El lenguaje de programación seleccionado para el proyecto de investigación está optimizado para la web, una de las posibles implicaciones del trayecto de investigación podría reforzar la presencia de las tecnologías y los planteamientos políticos de las redes distribuidas. 

La relación entre práctica e investigación artística apunta a la poneración entre estos dos rubros. El lado tecnológico de la investigación, aquel que implica la escritura de software, podría implicar una inmersión al bajo nivel de programación. Esta acción permitiría resolver aspectos que \textit{frameworks} dedicados no resuelven o lo hacen parcialmente. 

Agenda pendiente.

La escritura de y con software puede resolverse a un nivel mucho más comprometido que el trabajo con repositorios Git como GitHub.

Documentación que depende de empresas que pueden desaparecer o cambiar de perfil. 

Síntesis en la nube 
