\documentclass[10pt,letterpaper, twocolumn]{article}
\usepackage[utf8]{inputenc}
\usepackage[spanish]{babel}
\usepackage{endnotes}
\usepackage{hyperref}
\usepackage{amsmath}
\usepackage{amsfonts}
\usepackage{amssymb}
\usepackage{graphicx}
\usepackage{setspace}
\usepackage[sort&compress]{natbib}

% \onehalfspacing

\renewcommand{\notesname}{Notas}
\urlstyle{same}

\let\footnote=\endnote

\providecommand{\keywords}[1]
{
  \small	
  \textbf{\textit{Palabras clave---}} #1
}

\title{%
  Tres Estudios Abiertos \\
  \large Prácticas performáticas, audiovisuales y experimentales en el navegador}

\author{Emilio Ocelotl}
\begin{document}

\maketitle

\begin{abstract}

\textit{Tres Estudios Abiertos} es una investigación que aborda prácticas artísticas, performáticas, experimentales y audiovisuales para el navegador. Estudia la influencia de los lenguajes de programación y el aporte que la programación estética puede realizar a la investigación artística con tecnología. Como parte del objetivo tecnológico, busca implementar ideas sobre granulación a un esqueleto tecnológico y la escritura de módulos personalizados de software que conformarán una colección de estudios para el navegador. 

%Atiende a niveles altos de programación para el control de software por medio de una interfaz de texto. 

Estas piezas tendrán audio y video, estarán alojadas en la web, tendrán una lógica distribuida, estarán optimizadas para el bajo consumo de recursos de la computadora y no requerirán instalación ni dependencias adicionales. \textit{Javascript} es el lenguaje de programación principal del proyecto y estará presente en el nivel bajo, medio y alto de programación. % Posibilidad de enunciar otros lenguajes de programación ? 

El núcleo de la investigación será el bucle que retroalimenta práctica artística y reflexión, es decir, la relación que existirá entre las piezas realizadas, el entramado de software que posibilitará su realización en la web y las implicaciones tecno-sociales de la tecnología principal de este proyecto: los lenguajes de programación. 

\end{abstract}


\section*{Antecedentes}

% Artículo Panorama como antecedente del proyecto

Los antecedentes de este proyecto describen la transición del desarrollo de software para la realización de sistemas interactivos a la escritura de módulos de software audiovisuales. Estas experiencias parten de la optimización y la ligereza del lado del hardware (por ejemplo, con el uso de computadoras de placa reducida como Raspberry Pi o Jetson Nano ) y de sistemas ligeros, con pocas capas de abstracción y accesibles de síntesis y renderización de audio y video en el navegador.

Parte de los antecedentes están directamente relacionados con la experiencia performática de escribir de código al vuelo con fines creativos, audiovisuales y experimentales, de manera similar a como lo describen \cite{villasenor} para los casos de Barcelona y Ciudad de México. 

% De estas ideas se pueden extender reflexiones y puestas en práctica de ideas políticas relacionadas con la opotimización de recursos para el bajo consumo de energía y la resistencia a la obsolescencia programada de los dispositivos electrónicos. 

De manera directa, los siguientes proyectos están relacionados con la investigación y se convierten en casos de estudio, en algunas otras ocasiones fungen como antecedentes directos:

\begin{itemize}

\item Caso de estudio. THREE.studies \citep{threestudies} y en específico, las iteraciones threecln \citep{threecln} y threeBEASTs \citep{threeBEASTs}
\item Caso de estudio. Anti \citep{anti} 
\item Antecedente directo. Diálogo IV @ Coloquio Salvador Contreras. Artes Sonoras y Creación Musical en México: Siglo XXI \citep{dialogo}
\item Antecedente directo. Panorama. Escritura de espacios libres e inmersivos para el performance audiovisual \citep{PanoramaArticulo}\footnote{Una versión en construcción y en línea del artículo se puede encontrar en \url{https://piranhalab.github.io/panorama/}}
  
\end{itemize}

% De manera indirecta 

Algunos proyectos similares al que describe \textit{Tres Estudios Abiertos} son: 

\begin{itemize}

\item \textbf{Nivel Bajo:} Ruffbox \citep{ruffbox}, WebAssembly/Rust Tutorial\footnote{\url{https://www.toptal.com/webassembly/webassembly-rust-tutorial-web-audio} (Consultado el \today)} y Flocking \citep{flocking}.
\item \textbf{Nivel Medio:} \citep{supercolliderweb}, Web Audio API\footnote{\url{https://developer.mozilla.org/es/docs/Web/API/Web_Audio_API}}, Tone.js\footnote{\url{https://tonejs.github.io/}} y supercolliderjs\footnote{\url{https://github.com/crucialfelix/supercolliderjs/}}.
\item \textbf{Nivel Alto:} Estuary \citep{estuary}, Troop \citep{Troop}, flok \citep{flok}, tilt \citep{tilt}, LiveLab \citep{livelab}, Hydra \citep{hydra}, timeNot \citep{timenot} y seis8s \citep{seis8s}
  
\end{itemize}

 %% Esta es la parte más fuerte de la investigación
\section*{Plantamiento del problema}

De entre los proyectos similares destacamos aquellos que son de Nivel Alto para responder a la pregunta: ¿Cuál es la diferencia entre los proyectos mencionados y \textit{Tres Estudios Abiertos}? Dos perspectivas podrían aclarar el punto de partida del proyecto. La primera es funcional y hace referencia a la solución de problemas partiendo de una comunidad que ejecuta, retroalimenta y enriquece al proyecto. La segunda, apuesta por la diversidad en el desarrollo de interfaces de control y que se manifiesta de una manera muy específica en el proyecto Estuary. El presente proyecto busca responder en un momento anterior a la realización de módulos de software si hay diferencias estéticas heredadas de notaciones musicales y computacionales, lenguajes de programación,  estilos musicales, flujos de voltaje que desembocan en síntesis de audio / imagen e incluso planteamientos críticos sobre decolonialidad. Referimos a este conjunto de diferencias como decisiones de diseño en sintaxis de control que tienen consecuencias en la estética que resulta de controlar motores de audio y video. En este sentido, \textit{Tres Estudios Abiertos} es una búsqueda que orbita en estas desiciones y se adscribe a la diversidad en la escritura de y con software. 
 
\section*{Avances y Práctica}

El contexto de la pandemia de COVID-19 ha influído directamente en la investigación. La realización de piezas para el navegador que puedan responder a premisas bajo la restricción del distanciamiento social ha sido un reto para el momento de arranque de \textit{Tres Estudios Abiertos}. 

Actualmente el proyecto está compuesto de algunos módulos escritos. Algunas instancias se han desprendido los módulos y las primeras aproximaciones a la realización de obras audiovisuales para el navegador.

THREE.studies es una pieza realizada en el marco del programa Resilencia Sonora de Música UNAM. La interpretación (cello eléctrico) en todos los casos estuvo a cargo de Iracema de Andrade. La pieza contó con el apoyo estético y logístico de PiranhaLab (Marianne Teixido y Dorian Sotomayor)- 

Al momento de escritura del presente proyecto la primera instancia del estudio no ha sido estrenada.

Una versión fija ha sido exprandida y presentada en el marco de  BEAST FEaST 2021\footnote{\url{http://www.beast.bham.ac.uk/beast-feast-2021/online-works/}}.

Tras bambalinas el proyecto implicó un proceso de producción y montaje a distancia. Descripción. 

4NT1 y algunos eventos que abonan desde el contexto de la pieza. 

Además de las instancias prácticas del proyecto, la investigación se ha retroalimentado a partir de dos proyectos coexistentes: el Seminario Permamente de Tecnología Músical de la FaM y el texto que el colectivo \textit{PiranhaLab} escribe para la realización de este espacio.

 % Funcionaría como resultados
\section*{Conclusiones}

Las restricciones de copresencia del contexto pandémico ha posibilitado exploraciones superficiales que fungen como una observación de la panorámica de los rumbos de la investigación. Consideramos que la investigación tiene momentos abiertos que se resuelven en la realización creativa de piezas. La parte más robusta de la investigación se concentra en los antecedentes y en la detección de proyectos cercanos que delimitan al proyecto en términos de los alcances y retos, tomando en consideración un ecosistema relativamente activo de escritura y uso de software en la web potenciado en el contexto de la pandemia. 

Los avances actuales apuntan a la realización de piezas que hagan uso de sintetizadores granulares audiovisuales en la nube. Esto permite desplazar la carga de procesamiento de audio e imagen a un servidor dedicado o a un conjunto de computadoras conectadas entre sí. Este aspecto permitiría realizar análisis del flujo de audio e imagen detallados sin que esto represente un problema para los recursos de las máquinas que acceden a las piezas y los espacios. Esta solución también podría implicar una alternativa para aprovechar las posibilidades de motores gráficos, de audio y de análisis de información mucho más robustos que aquellos que actualmente están disponibles para el navegador y que funcionan únicamente de forma local. La centralidad del procesamiento de señales y la omisión de la experiencia de usuario ha sido una crítica señalada. 

%El lenguaje de programación seleccionado para el proyecto de investigación está optimizado para la web, una de las posibles implicaciones del trayecto de investigación podría reforzar la presencia de las tecnologías y los planteamientos políticos de las redes distribuidas. 

La presente investigación ha tenido en cuenta la ponderación entre práctica e investigación artística. El lado tecnológico de la investigación, aquel que implica la escritura de software, podría orientarse al bajo nivel de programación. Esta acción permitiría resolver aspectos que \textit{frameworks} dedicados resuelven parcialmente. Sobre la realización de la investigación, destacamos la importancia del bucle y la metafora como estrategias para la observación y la práctica transversal que se retroalimenta mientras se ejecuta. 

% Agenda pendiente.

El epicentro de la escritura podría recaer en el mismo texto/código y no en plataformas de repositorios que pertenecen a empresas que pueden cambiar de giro en cualquier momento. El respaldo del trabajo por medios digitales podría encontrar soluciones actuales en la encriptación. Ambas posibilidades pueden extender la escritura a formas experimentales e integradas al código. 

Como parte de las consecuencias señaladas y no buscadas del proyecto se encuentran el activismo ecológico y los planteamientos éticos del software. Estos dos aspectos no son necesariamente centrales pero pueden tenerse en cuenta dentro de una agenda de acción que se desborda de la práctica artística y la investigación.  



%% ¿Qué? En el otro docu esto es el planteamiento del problema 
%% ¿Cómo? Metodología 
%% ¿Por qué? Para qué? Igual y esto podría ir en las conclusiones 
%% Conclusiones 

\end{document}
