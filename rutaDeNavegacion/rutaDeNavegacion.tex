% Tres-Estudios-Abiertos. Ruta de navegación  

\documentclass[12pt,letterpaper, twocolumn]{article}
\usepackage[utf8]{inputenc}
\usepackage[spanish, es-notilde]{babel}
\usepackage{endnotes}
\usepackage{hyperref}
\usepackage{amsmath}
\usepackage{amsfonts}
\usepackage{amssymb}
\usepackage[sort&compress]{natbib}

\renewcommand{\notesname}{Notas}

\hypersetup{
    colorlinks=true,
    linkcolor=blue,
    filecolor=magenta,      
    urlcolor=magenta,
    citecolor=magenta,
}

\urlstyle{same}

\author{Emilio Ocelotl}
%\title{Panorama. Tecnologías libres e inmersivas para el performance audiovisual}

\title{%
  Tres Estudios Abiertos \\
  \large Nuevas prácticas performáticas audiovisuales experimentales para el navegador.}

\providecommand{\keywords}[1]
{
  \small	
  \textbf{\textit{Palabras clave---}} #1
}

\let\footnote=\endnote

\begin{document}

\maketitle

\begin{abstract}
  
\textit{Tres Estudios Abiertos} es una investigación que aborda prácticas artísticas, performáticas, experimentales y audiovisuales para el navegador. Estudia la influencia de los lenguajes de programación y el aporte que la programación estética puede realizar a la investigación artística con tecnología. Como parte del objetivo tecnológico, busca implementar ideas sobre granulación a un esqueleto tecnológico y la escritura de módulos personalizados de software que conformarán una colección de estudios para el navegador. 

%Atiende a niveles altos de programación para el control de software por medio de una interfaz de texto. 

Estas piezas tendrán audio y video, estarán alojadas en la web, tendrán una lógica distribuida, estarán optimizadas para el bajo consumo de recursos de la computadora y no requerirán instalación ni dependencias adicionales. \textit{Javascript} es el lenguaje de programación principal del proyecto y estará presente en el nivel bajo, medio y alto de programación. % Posibilidad de enunciar otros lenguajes de programación ? 

El núcleo de la investigación será el bucle que retroalimenta práctica artística y reflexión, es decir, la relación que existirá entre las piezas realizadas, el entramado de software que posibilitará su realización en la web y las implicaciones tecno-sociales de la tecnología principal de este proyecto: los lenguajes de programación. 

\end{abstract}


\keywords{software, navegador, javascript, granulación, cyberespacio, par-a-par, programación estética, distribución, webAssembly, webGL}


% \section{Ecosistema}

% Articulación con una propuesta de investigación que quedo deshechada. ¿Es posible potenciar los proyectos en un contexto social, es decir, en una dimensión que rebasa la propia obra (y por lo tanto el individuo) ? 

Ecosistema como una forma de nombrar al entramado de Nodos y Trayectos. 

¿En qué contexto se inserta mi proyecto ? Impulso de desarrollos o DWO

Un referente que no necesariamente sea una pieza artística pero que genere condiciones para la práctica. 
 % Introducción, antecedentes, proyectos y tecnologías similares.

% 1. Antecedentes de la investigación y marco teórico
% 2. Planteamiento del problema
% 3. Proposición o hipótesis 
% 4. Justificación
% 5. Objetivos y metas correspondientes
% 6. Cronograma


\theendnotes % El artículo usa notas al final del texto. Aquí aparecen hipervínculos y conceptos de apoyo. Conceptos 

%\nocite{interactiveDigitalMusic}
% \nocite{speakingCode} 

\addcontentsline{toc}{chapter}{\protect\numberline{}Referencias}%
\bibliography{bib/panBib}{} % Referencias bibliográficas y repositorios en Git como referencias en la web.
\bibliographystyle{apalike-es}

\end{document}
