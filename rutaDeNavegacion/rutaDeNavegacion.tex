% Tres-Estudios-Abiertos. Ruta de navegación  

\documentclass[11pt,letterpaper, twocolumn]{article}
\usepackage[utf8]{inputenc}
\usepackage[spanish, es-notilde]{babel}
\usepackage{endnotes}
\usepackage{hyperref}
\usepackage{amsmath}
\usepackage{amsfonts}
\usepackage{amssymb}
\usepackage{graphicx}
\usepackage{titling}
\usepackage{lmodern}
\usepackage[sort&compress]{natbib}

\pretitle{%
  \begin{center}
    \LARGE
    
  \includegraphics[width=\textwidth]{../img/bannerPrincipal.png}\\[\bigskipamount]
}
\posttitle{\end{center}}


\renewcommand{\notesname}{Notas}

\hypersetup{
    colorlinks=true,
    linkcolor=blue,
    filecolor=magenta,      
    urlcolor=magenta,
    citecolor=magenta,
}

\urlstyle{same}

\author{Emilio Ocelotl}
%\title{Panorama. Tecnologías libres e inmersivas para el performance audiovisual}

\title{%
  Tres Estudios Abiertos \\
  \large Nuevas prácticas performáticas audiovisuales experimentales para el navegador.}

\providecommand{\keywords}[1]
{
  \small	
  \textbf{\textit{Palabras clave---}} #1
}

\let\footnote=\endnote

\begin{document}

\maketitle


\textit{Tres Estudios Abiertos} es una investigación que aborda prácticas artísticas, performáticas, experimentales y audiovisuales para el navegador. Estudia la influencia de los lenguajes de programación y el aporte que la programación estética puede realizar a la investigación artística con tecnología. Como parte del objetivo tecnológico, busca implementar ideas sobre granulación a un esqueleto tecnológico y la escritura de módulos personalizados de software que conformarán una colección de estudios para el navegador. 

%Atiende a niveles altos de programación para el control de software por medio de una interfaz de texto. 

Estas piezas tendrán audio y video, estarán alojadas en la web, tendrán una lógica distribuida, estarán optimizadas para el bajo consumo de recursos de la computadora y no requerirán instalación ni dependencias adicionales. \textit{Javascript} es el lenguaje de programación principal del proyecto y estará presente en el nivel bajo, medio y alto de programación. % Posibilidad de enunciar otros lenguajes de programación ? 

El núcleo de la investigación será el bucle que retroalimenta práctica artística y reflexión, es decir, la relación que existirá entre las piezas realizadas, el entramado de software que posibilitará su realización en la web y las implicaciones tecno-sociales de la tecnología principal de este proyecto: los lenguajes de programación. 


\keywords{software, navegador, javascript, granulación, cyberespacio, par-a-par, programación estética, distribución, webAssembly, webGL}
 

\section*{Antecedentes}

% Artículo Panorama como antecedente del proyecto

Los antecedentes de este proyecto describen la transición del desarrollo de software para la realización de sistemas interactivos a la escritura de módulos de software audiovisuales. Estas experiencias parten de la optimización y la ligereza del lado del hardware (por ejemplo, con el uso de computadoras de placa reducida como Raspberry Pi o Jetson Nano ) y de sistemas ligeros, con pocas capas de abstracción y accesibles de síntesis y renderización de audio y video en el navegador.

Parte de los antecedentes están directamente relacionados con la experiencia performática de escribir de código al vuelo con fines creativos, audiovisuales y experimentales, de manera similar a como lo describen \cite{villasenor} para los casos de Barcelona y Ciudad de México. 

% De estas ideas se pueden extender reflexiones y puestas en práctica de ideas políticas relacionadas con la opotimización de recursos para el bajo consumo de energía y la resistencia a la obsolescencia programada de los dispositivos electrónicos. 

De manera directa, los siguientes proyectos están relacionados con la investigación y se convierten en casos de estudio, en algunas otras ocasiones fungen como antecedentes directos:

\begin{itemize}

\item Caso de estudio. THREE.studies \citep{threestudies} y en específico, las iteraciones threecln \citep{threecln} y threeBEASTs \citep{threeBEASTs}
\item Caso de estudio. Anti \citep{anti} 
\item Antecedente directo. Diálogo IV @ Coloquio Salvador Contreras. Artes Sonoras y Creación Musical en México: Siglo XXI \citep{dialogo}
\item Antecedente directo. Panorama. Escritura de espacios libres e inmersivos para el performance audiovisual \citep{PanoramaArticulo}\footnote{Una versión en construcción y en línea del artículo se puede encontrar en \url{https://piranhalab.github.io/panorama/}}
  
\end{itemize}

% De manera indirecta 

Algunos proyectos similares al que describe \textit{Tres Estudios Abiertos} son: 

\begin{itemize}

\item \textbf{Nivel Bajo:} Ruffbox \citep{ruffbox}, WebAssembly/Rust Tutorial\footnote{\url{https://www.toptal.com/webassembly/webassembly-rust-tutorial-web-audio} (Consultado el \today)} y Flocking \citep{flocking}.
\item \textbf{Nivel Medio:} \citep{supercolliderweb}, Web Audio API\footnote{\url{https://developer.mozilla.org/es/docs/Web/API/Web_Audio_API}}, Tone.js\footnote{\url{https://tonejs.github.io/}} y supercolliderjs\footnote{\url{https://github.com/crucialfelix/supercolliderjs/}}.
\item \textbf{Nivel Alto:} Estuary \citep{estuary}, Troop \citep{Troop}, flok \citep{flok}, tilt \citep{tilt}, LiveLab \citep{livelab}, Hydra \citep{hydra}, timeNot \citep{timenot} y seis8s \citep{seis8s}
  
\end{itemize}


\section*{Plantamiento del problema}

De entre los proyectos similares destacamos aquellos que son de Nivel Alto para responder a la pregunta: ¿Cuál es la diferencia entre los proyectos mencionados y \textit{Tres Estudios Abiertos}? Dos perspectivas podrían aclarar el punto de partida del proyecto. La primera es funcional y hace referencia a la solución de problemas partiendo de una comunidad que ejecuta, retroalimenta y enriquece al proyecto. La segunda, apuesta por la diversidad en el desarrollo de interfaces de control y que se manifiesta de una manera muy específica en el proyecto Estuary. El presente proyecto busca responder en un momento anterior a la realización de módulos de software si hay diferencias estéticas heredadas de notaciones musicales y computacionales, lenguajes de programación,  estilos musicales, flujos de voltaje que desembocan en síntesis de audio / imagen e incluso planteamientos críticos sobre decolonialidad. Referimos a este conjunto de diferencias como decisiones de diseño en sintaxis de control que tienen consecuencias en la estética que resulta de controlar motores de audio y video. En este sentido, \textit{Tres Estudios Abiertos} es una búsqueda que orbita en estas desiciones y se adscribe a la diversidad en la escritura de y con software. 

\section*{Marco Teórico}

La presente investigación parte del giro de los nuevos medios y de los estudios del software \citep{manovichlanguage}. Como una extensión del punto de partida, la investigación se adscribe a la escritura con y sobre software \citep{aestheticProgramming}. \textit{Tres Estudios Abiertos} desemboca en el papel que juega la experiencia subjetiva y las implicaciones políticas y sociales en la programación estética \citep{speakingCode}. 


\section*{Premisa y objetivos}

% Partimos del bucle de investigación-creación para reducir la brecha de la escritura y la investigación sobre y con software. > Esto va en metodología según yo 

La premisa principal principal de la investigación consiste en estudiar casos específicos de programación estética orientada a la integración audiovisual, posibilitada a partir de lenguajes de programación, específicamente Javascript, en el contexto del bucle investigación-creación. 

De esta premisa se desprenden los objetivos secundarios: escribir software que pueda ser implementado en el contexto de obras audiovisuales para el navegador y realizar reflexiones a manera de documentación que vinculen conceptos tecno-sociales y estéticos. 


\section*{Metodología}

%¿Cómo responder a la hipótesis principal? ¿Es posible medir de alguna manera la experiencia estética y subjetiva de un lenguaje de programación? ¿Es posible encontrar una respuesta en el alto nivel? 

La presente investigación hace referencia a la investigación artística y en específico retoma la idea de \textit{loop} o bucle para hablar de la relación entre investigación-creación, tal y como se expresa en \citep{ocelotlMas}. Esta forma de trabajo retroalimenta la escritura investigativa con la práctica artística y de regreso. Consideramos que esta forma de proceder en la investigación revela aspectos que el distanciamiento convencional en la investigación no toma en cuenta.

En esta investigación hacemos referencia a procesos artísticos que involucran música, software y computadoras. En este sentido, las peculiaridades de la investigación se toman en cuenta para plantear un proyecto con escritura de software. El objetivo primordial del proyecto no busca realizar mediciones para establecer parámetros de ligereza o eficacia el software resultante. En todo caso busca aprovechar la lógica tecnológica de la programación y del procesamiento de información para encontrar soluciones para la investigación y la reflexión.

Uno de los aspectos que más rescatamos en este sentido es el uso del repositorio como una estrategia para trabajar con la escritura de software, un respaldo para el trabajo colaborativo y como una documentación que puede modificarse y consultarse en el tiempo. Para la presente investigación la noción de repositorio de código es central ya que nos permite adentarnos en \textit{las tripas del software} al mismo tiempo que atiende al código como un recurso de investigación. En este sentido la escritura y la consulta de software puede realizarse con sistemas distribuidos de control de versiones como \textit{Git}\footnote{``\textit{Git} es un sistema distribuido de control de versiones libre y abierto diseñado para tratar con todo, desde proyectos pequeños hasta proyectos muy grandes con velocidad y eficiencia.'' \url{https://git-scm.com/} (Consultado el \today)}. De esta manera buscamos lidiar con el caracter efímero del software y de piezas artísticas para el navegador. 

Aprovechamos la lógica de los sistemas distribuidos para el control de versiones para construir un entramado que pueda dar cuenta, por un lado, del proceso creativo con los respositorios de las piezas que existen en la web y por el otro, el texto que conforma la investigación y que pretende discutir con la contraparte de programación. De esta manera los procesos quedan lo suficientemente abiertos para interrelacionarse sin perder delimitación y diferenciación. En este sentido buscamos extender la propuesta de un momento anterior de investigación que anuncia la lógica del trabajo con sistemas distribuidos para la investigación artística con tecnología.

La ejecución de la investigación consiste en relacionar, código, texto y recursos multimedia. En un momento previo de decisión el proyecto ponderará la implementación de distintos sistemas de escritura de texto/código y elegiremos el que mejor se adapte a los alcances del proyecto. Hasta el momento la investigación considera tres sistemas que de alguna u otra manera se relacionan con la escritura de texto/código: \textit{LaTeX}\footnote{LaTeX es un sistema de composición tipográfica de alta calidad; incluye funcionalidades diseñadas para la producción de documentación técnica y científica. \url{https://www.latex-project.org/} (Consultado el \today)}, \textit{Git} y \textit{JupyterLab}\footnote{\textit{JupyterLab} es un entorno de desarrollo interactivo basado en la web para \textit{notebooks} de \textit{Jupyter}, código y datos. \url{https://jupyter.org/}. (Consultado el \today)}. La elección o combinación de entornos para la escritura de la tesis requerirán una ponderación que tome en cuenta la integración entre texto, código y acoplamiento con el lenguaje de programación principal del proyecto: \textit{Javascript}. De manera complementaria, la investigación toma en consideración los alcances de diseño, composición tipográfica y estilo personalizados. Como una alternativa al formato de presentación impreso (digital o físico), el proyecto busca que el proceso y el resultado de la investigación pueda ser compilado y consultado en línea como una página web.\footnote{Una prueba de esta propuesta se puede consultar en: \url{https://emilioocelotl.github.io/tres-estudios-abiertos/}}

La ejecución de la investigación coincide con los planteamientos y la delimitación del proceso de reflexión-creación: estos procesos implican múltiples objetivos que corren al mismo tiempo y que podemos enunciar de manera general como: tecnológico, estético y de investigación. El uso de conceptos que atraviesen estos rubros nos permitirá desplazarnos a partir de la retroalimentación que se genera entre tecnología y propuesta artística. El vínculo hacia lo reflexivo puede establecerse en las plataformas para escribir código como entornos de desarrollo integrados (IDEs) y sistemas de control de versiones, en la escritura por sí misma como tecnología, en el uso de conceptos y su vinculación con la metáfora. 


% \section*{Justificacion}

Problematizar el software desde lo político y en la escalada de recursos para ejecutar un programa. La escritura de programas de bajo nivel como una declaratoria de motivaciones adicionales a la resolución funcional de un programa. 

El punto de partida puede ser la escritura de software para la investigación. 

Esto puede implicar aspectos relacionados con uso de pocos recursos energéticos y de hardware. La discusión puede llevarse hacía la inclusión / exclusión de experiencias. De manera indirecta puede plantearse la crítica a la industria del software como los mayores escaladores de recursos.


\section*{Objetivos}

% Audio e imagen o audiovisual peso de lo que voy a hacer 

\textbf{Objetivo principal:} Explorar el aporte estético a la práctica artística de lenguajes de programación para la generación de sonido e imagen en el navegador y el aporte que éstos pueden realizar a la investigación artística que involucra tecnología.

\textbf{Objetivos secundarios:} 1) Realizar reflexiones a manera de documentación que vinculen a través de la metáfora conceptos tecno-sociales y estéticos, 2) implementar el entramado de módulos en obras audiovisuales y 3) diseñar una serie de funciones personalizadas como interfaz textual que puedan mediar entre niveles de programación y que discutan con el contexto social en el que se enmarcan.

% instrucciones > funciones

% Processing microdifererencias como si fuera un dialecto de java 

% Escribir una serie de módulos interconectados de software para realizar sintesis granular audiovisual en el navegador.

%\section*{Objetivos Secundarios}

%\begin{itemize}

%\item Realizar reflexiones a manera de documentación que vinculen a través de la metáfora conceptos tecno-sociales y estéticos
%\item Implementar el entramado de módulos en obras audiovisuales. 
%\item Utilizar una serie de instrucciones personalizadas como interfaz textual que puedan mediar entre niveles y que discutan con el contexto cultural en el que se enmarcan.
% \item Aportar módulos de desarrollo tecnólogico que puedar ser de utilidad para otros proyectos.
% \item Estudiar estrategias de colaboración y creación a distancia, por medio de la computadora y e instrumentos acústicos.
% \item Comparar críticamente el desarrollo tecnológico y la tecnología existente.
% \item Documentar los desarrollos y resultados secundarios realizados en colectivo.

%\end{itemize}

% \section*{Estructura sugerida}

\begin{itemize}

\item Resumen y palabras clave.
\item Introducción. Antecedentes,  motivaciones e hipótesis/premisa (¿Qué?).
\item Metodología. ¿Cómo se responde a la hipótesis/premisa?
\item Resultados. Descripción de piezas y entramados de código.
\item Discusión. A partir de los resultados ¿Qué podemos aportar?
\item conclusión. ¿Se demostró la hipótesis/premisa? 
  
\end{itemize}

% \section*{Cronograma}

Primer semestre - Realización de maqueta 1

\begin{itemize}

\item 12.02.21 - Entrega final Música UNAM - threecln
\item 25.02.21 - Delimitación de cara a una propuesta para presentaciones en vivo.
\item S/Fecha - Presentación Música UNAM. Mediados de año

\item 01.03.21 - Fecha límite para tener y enviar el resumen
\item 22.07.21 - Entrega final para seminario de Tecnología Musical

\item 02.03.21 - Primer informe
\item 18.03.21 - Primer encuentro
\item 19.10.21 - Informe final

 \end{itemize}

% \section{Ecosistema}

% Articulación con una propuesta de investigación que quedo deshechada. ¿Es posible potenciar los proyectos en un contexto social, es decir, en una dimensión que rebasa la propia obra (y por lo tanto el individuo) ? 

Ecosistema como una forma de nombrar al entramado de Nodos y Trayectos. 

¿En qué contexto se inserta mi proyecto ? Impulso de desarrollos o DWO

Un referente que no necesariamente sea una pieza artística pero que genere condiciones para la práctica. 
 % Introducción, antecedentes, proyectos y tecnologías similares.

% 1. Antecedentes de la investigación y marco teórico
% 2. Planteamiento del problema
% 3. Proposición o hipótesis
% 4. Metodo
% 5. Justificación
% 6. Objetivos y metas correspondientes
% 7. Estructura general sugerida
% 8. Cronograma


\theendnotes % El artículo usa notas al final del texto. Aquí aparecen hipervínculos y conceptos de apoyo. Conceptos 

%\nocite{interactiveDigitalMusic}
% \nocite{speakingCode} 

\addcontentsline{toc}{chapter}{\protect\numberline{}Referencias}%
\bibliography{bib/panBib}{} % Referencias bibliográficas y repositorios en Git como referencias en la web.
\bibliographystyle{apalike-es}

\end{document}
