% Tres-Estudios-Abiertos. Ruta de navegación  

\documentclass[12pt,letterpaper, twocolumn]{article}
\usepackage[utf8]{inputenc}
\usepackage[spanish, es-notilde]{babel}
\usepackage{endnotes}
\usepackage{hyperref}
\usepackage{amsmath}
\usepackage{amsfonts}
\usepackage{amssymb}
\usepackage[sort&compress]{natbib}

\renewcommand{\notesname}{Notas}

\hypersetup{
    colorlinks=true,
    linkcolor=blue,
    filecolor=magenta,      
    urlcolor=magenta,
    citecolor=magenta,
}

\urlstyle{same}

\author{Emilio Ocelotl}
%\title{Panorama. Tecnologías libres e inmersivas para el performance audiovisual}

\title{%
  Tres Estudios Abiertos \\
  \large Nuevas prácticas performáticas audiovisuales experimentales para el navegador.}

\providecommand{\keywords}[1]
{
  \small	
  \textbf{\textit{Palabras clave---}} #1
}

\let\footnote=\endnote

\begin{document}

\maketitle

\begin{abstract}
  
\section*{Resumen}

\textit{Tres Estudios Abiertos} es un proyecto de investigación que busca adentrarse en nuevas prácticas experimentales y audiovisuales para el navegador. Parte de la programación de un esqueleto de granulación audiovisual complementado con módulos personalizados de software. Atiende a niveles altos de programación para el control de software por medio de una interfaz de texto. 

Toma en cuenta piezas para el navegador que fungen casos de estudio. El proyecto busca demostrar que las lógicas de los lenguajes de programación posibilitan formas de pensamiento específicos con consecuencias estéticas definidas. 

Los casos de estudio son entramados de audio y pretenden ser ligeros, sin instalaciones, sin referencias a instalación de dependencias de terceros, basados en la web, distribuidos y optimizados para el bajo consumo de recursos de la computadora. 

Hasta el momento considera a \textit{Javascript}\footnote{¿Todavía es posible usar otros lenguajes de programación?} como el lenguaje principal de los casos de estudio. De manera secundaria el proyecto se pregunta si es posible deducir ideas estéticas de las arquitecturas de los lenguajes de programación para la escritura de interfaces textuales de control. 

Adicionamente al estético y tecnológico, consideramos un tercer hilo que corre paralelamente al bucle de investigación-creación: el investigativo. En esta pista están presentes reflexiones e implicaciones políticas en la escritura de y con software.

La investigación busca relacionar la escritura del texto resultante con la escritura de software y otros medios (principalmente repositorios de código pero también imágenes, sitios web, sonido). De esta manera el documento resultante busca desbordarse del papel y de la palabra escrita. 

\end{abstract}


\keywords{navegador, cyberespacio, p2p}


% \section{Ecosistema}

Ecosistema como una forma de nombrar al entramado de Nodos y Trayectos. 

¿En qué contexto se inserta mi proyecto ? Impulso de desarrollos o DWO

Un referente que no necesariamente sea una pieza artística pero que genere condiciones para la práctica. 

\subsection{Contexto}
\subsection{Planteamiento Inicial}
\subsection{Proceso y realización}



\subsection{Resultados}
 % Introducción, antecedentes, proyectos y tecnologías similares.

% 1. Antecedentes de la investigación y marco teórico
% 2. Planteamiento del problema
% 3. Proposición o hipótesis 
% 4. Justificación
% 5. Objetivos y metas correspondientes
% 6. Cronograma


\theendnotes % El artículo usa notas al final del texto. Aquí aparecen hipervínculos y conceptos de apoyo. Conceptos 

%\nocite{interactiveDigitalMusic}
% \nocite{speakingCode} 

\addcontentsline{toc}{chapter}{\protect\numberline{}Referencias}%
\bibliography{bib/panBib}{} % Referencias bibliográficas y repositorios en Git como referencias en la web.
\bibliographystyle{apalike-es}

\end{document}
