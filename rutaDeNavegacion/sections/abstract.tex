
\section*{Resumen}

\textit{Tres Estudios Abiertos} es un proyecto de investigación que busca adentrarse en nuevas prácticas experimentales y audiovisuales para el navegador.

Toma en cuenta casos de estudio para demostrar que las lógicas de los lenguajes de programación posibilitan formas de pensamiento específicos y distintos entre sí en la conformación de software orientado al performance audiovisual y a la programación de obras fijas.

Las condiciones iniciales del proyecto buscan que el entramado de software sea ligero, sin instalaciones, sin referencias a instalación de dependencias de terceros, basado en la web, distribuido y optimizado para el bajo consumo de recursos de la computadora. 

Hasta el momento considera a \textit{Javascript}\footnote{¿Todavía es posible usar otros lenguajes de programación?} como el lenguaje principal de los casos de estudio. De manera secundaria el proyecto se pregunta si es posible deducir ideas estéticas de las arquitecturas de los lenguajes de programación para la escritura de interfaces textuales de control. 

Adicionamente al estético y tecnológico, consideramos un tercer hilo que corre paralelamente: el investigativo. En esta pista están presentes reflexiones e implicaciones políticas en la escritura de y con software. 

Finalmente, \textit{Tres Estudios Abiertos} busca relacionar la escritura del texto resultante con la escritura de módulos de software y otros medios (principalmente repositorios de código pero también imágenes, sitios web, sonido). De esta manera el documento resultante busca desbordarse del papel y de la palabra escrita. 
