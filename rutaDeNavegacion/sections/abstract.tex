
\textit{Tres Estudios Abiertos} es una reflexión sobre prácticas experimentales y audiovisuales para el navegador. Explora la influencia de los lenguajes de programación en la práctica artística y el aporte que pueden realizar a la investigación artística con tecnología. Implementa un esqueleto de granulación audiovisual y módulos personalizados de software que conformarán una colección de estudios para el navegador. 

%Atiende a niveles altos de programación para el control de software por medio de una interfaz de texto. 

Estas piezas estarán alojadas en la web, serán distribuidas y optimizadas para el bajo consumo de recursos de la computadora y no requerirán instalación ni dependencias. El proyecto considera a \textit{Javascript} como el lenguaje de programación que permeará al proyecto y que permitirá reflexionar en el nivel bajo, medio y alto de programación. % Posibilidad de enunciar otros lenguajes de programación ? 

El foro de observación de la investigación no son las piezas realizadas o el entramado de software que posibilitan su ejecución en la web, tampoco es la reflexión sobre las implicaciones tecno-sociales de los lenguajes de programación. El núcleo que esta investigación busca observar es el bucle que retroalimenta práctica artística y reflexión. % Pero bien podría ser uno u otro lado, supongo que esto es parte de la ponderación a resolver. 
