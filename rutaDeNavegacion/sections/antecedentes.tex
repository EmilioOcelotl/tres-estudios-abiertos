
\section*{Antecedentes}

% Artículo Panorama como antecedente del proyecto

Los antecedentes de este proyecto describen la transición del desarrollo de software para la realización de sistemas interactivos a la escritura de módulos de software audiovisuales. Estas experiencias parten de la optimización y la ligereza del lado del hardware (por ejemplo, con el uso de computadoras de placa reducida como Raspberry Pi o Jetson Nano ) y de sistemas ligeros, con pocas capas de abstracción y accesibles de síntesis y renderización de audio y video en el navegador.

Parte de los antecedentes están directamente relacionados con la experiencia performática de escribir de código al vuelo con fines creativos, audiovisuales y experimentales. 

% De estas ideas se pueden extender reflexiones y puestas en práctica de ideas políticas relacionadas con la opotimización de recursos para el bajo consumo de energía y la resistencia a la obsolescencia programada de los dispositivos electrónicos. 

De manera directa, los siguientes casos están relacionados con la investigación y funcionan como antecedente:

\begin{itemize}

\item THREE.studies \citep{threestudies} y en específico, la iteración threecln \citep{threecln}
\item Diálogo IV
\item Anti
\item Panorama 
  
\end{itemize}

% De manera indirecta 

Algunos proyectos similares al que describe \textit{Tres Estudios Abiertos}.

\begin{itemize}

\item \textbf{Nivel Bajo:} \citep{ruffbox}, WebAssembly/Rust Tutorial\footnote{\url{https://www.toptal.com/webassembly/webassembly-rust-tutorial-web-audio} (Consultado el \today)} y \citep{flocking}.
\item \textbf{Nivel Medio:} \citep{supercolliderweb}, Web Audio API\footnote{\url{https://developer.mozilla.org/es/docs/Web/API/Web_Audio_API}}, Tone.js\footnote{\url{https://tonejs.github.io/}} y supercolliderjs\footnote{\url{https://github.com/crucialfelix/supercolliderjs/}}.
\item \textbf{Nivel Alto:} \citep{estuary}
  
\end{itemize}

¿Cuál es la diferencia entre los proyectos mencionados y \textit{Tres Estudios Abiertos}. 
