
\section*{Antecedentes y Marco Teórico}

% Artículo Panorama como antecedente del proyecto

Los antecedentes de este proyecto describen la transición del desarrollo de software para la realización de sistemas interactivos a la escritura de módulos de software audiovisuales. Estas experiencias parten de la optimización y la ligereza del lado del hardware (por ejemplo, con el uso de computadoras de placa reducida como Raspberry Pi o Jetson Nano ) y de sistemas ligeros y accesibles para la síntesis y renderización de audio y video en el navegador.

Parte de los antecedentes están directamente relacionados con la experiencia performática de escribir código al vuelo con fines creativos, audiovisuales y experimentales, tal y como lo describen \cite{villasenor} para los casos de Barcelona y Ciudad de México. 

La presente investigación parte del giro de los nuevos medios y de los estudios del software \citep{manovichlanguage}. Como una extensión del punto de partida, la investigación se adscribe a la escritura con y sobre software \citep{aestheticProgramming}. \textit{Tres Estudios Abiertos} desemboca en el papel que juega la experiencia subjetiva y las implicaciones políticas y sociales en la programación estética \citep{speakingCode}. 

\textit{Tres Estudios Abiertos} es parte de una trilogía de investigación que tiene dos momentos anteriores. Objeto, Paisaje y Efecto \citep{ocelotlLic} fue el primero de ellos. Es el texto resultante de un proyecto de investigación previo que abordó las nociones de objeto sonoro \citep{schaeffer}, paisaje sonoro\citep{schafer1} y efecto sonoro \citep{augoyard} para considerar a la escucha como un recurso para la investigación sociológica en música y par la investigación social desde el sonido. 

Un segundo punto de investigación refiere a un proyecto sobre tecnología musical \citep{ocelotlMas}. Este involucró un proceso de investigación-producción artística. La realización de este proyecto fue un prototipo tecnológico y cabe destacar que algunos aspectos que inicialmente estaban propuestos como secundarios pero que se revelaron como parte del núcleo en la investigación. Estos aspectos son: 1) el proceso de trabajo colaborativo, 2) la reflexión sobre la interacción entre audio e imagen en la composición musical electroacústica y 4) el uso de herramientas libres, personalizadas para la realización de prototipos audiovisuales y para el planteamiento de una observación crítica de los procesos creativos del mismo autor/compositor desde una perspectiva tecno-social. La propuesta de los estudios del software fue incorporada en momento de investigación. 

Otro antecedente de este proyecto es la práctica y reflexión planteada en colectivo por\textit{PiranhaLab}\footnote{``PiranhaLab es un laboratorio interdisciplinario que trabaja en las tripas del software''. \url{https://piranhalab.github.io/} (Consultado el \today)}. El ciclo de talleres realizado en el Centro de Cultura Digital (CCD) en coparticipación con el Laboratorio de Tecnologías Libres\footnote{Actualmente Laboratorio de Tecnologías Compartidas} permitió plantear dos conclusiones que permean a \textit{Tres Estudios Abiertos}: La difuminación de la distinción usuario/desarrollador como una motivación para la escritura de software y la diversidad en la escritura de software en América Latina.

La escritura de espacios para el ciclo de conciertos EDGES realizado por el Taller de Imágenes en Movimiento del Centro Multimedia (CMM) permitió explorar estas posibilidades en el contexto del encierro por la pandemia de COVID-19 y de espacios tridimensionales inmersivos en el navegador. Técnica y conceptualmente la escritura de estos espacios influye en el presente proyecto. Como un antecedente de investigación el artículo \textit{Panorama} \citep{panoramaArticulo} hace referencia al ecosistema de espacios y propuestas que también inciden en \textit{Tres Estudios Abiertos}.

% De estas ideas se pueden extender reflexiones y puestas en práctica de ideas políticas relacionadas con la opotimización de recursos para el bajo consumo de energía y la resistencia a la obsolescencia programada de los dispositivos electrónicos. 

De manera directa, los siguientes proyectos están relacionados con la investigación y se convierten en casos de estudio, en algunas otras ocasiones fungen como antecedentes directos:

\begin{itemize}

\item Caso de estudio. THREE.studies \citep{threestudies} y en específico, las iteraciones threecln \citep{threecln} y threeBEASTs \citep{threeBEASTs}
\item Caso de estudio. Anti \citep{anti} 
\item Antecedente directo. Diálogo IV @ Coloquio Salvador Contreras. Artes Sonoras y Creación Musical en México: Siglo XXI \citep{dialogo}
\item Antecedente directo. Panorama. Escritura de espacios libres e inmersivos para el performance audiovisual \citep{panoramaArticulo}\footnote{Versión en construcción y en línea: \url{https://piranhalab.github.io/panorama/}}
  
\end{itemize}

% De manera indirecta 

Algunos proyectos similares al que describe \textit{Tres Estudios Abiertos} son: 

\begin{itemize}

\item \textbf{Nivel Bajo:} Ruffbox \citep{ruffbox}, WebAssembly/Rust Tutorial\footnote{\url{https://www.toptal.com/webassembly/webassembly-rust-tutorial-web-audio} (Consultado el \today)} y Flocking \citep{flocking}.
\item \textbf{Nivel Medio:} \citep{supercolliderweb}, Web Audio API\footnote{\url{https://developer.mozilla.org/es/docs/Web/API/Web_Audio_API}}, Tone.js\footnote{\url{https://tonejs.github.io/}} y supercolliderjs\footnote{\url{https://github.com/crucialfelix/supercolliderjs/}}.
\item \textbf{Nivel Alto:} Estuary \citep{estuary}, Troop \citep{Troop}, flok \citep{flok}, tilt \citep{tilt}, LiveLab \citep{livelab}, Hydra \citep{hydra}, timeNot \citep{timenot} y seis8s \citep{seis8s}
  
\end{itemize}
