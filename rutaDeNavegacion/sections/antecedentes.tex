
\section*{Antecedentes}

% Artículo Panorama como antecedente del proyecto

Los antecedentes de este proyecto describen la transición del desarrollo de software para la realización de sistemas interactivos a la escritura de módulos de software audiovisuales. Estas experiencias parten de la optimización y la ligereza del lado del hardware (por ejemplo, con el uso de computadoras de placa reducida como Raspberry Pi o Jetson Nano ) y de sistemas ligeros, con pocas capas de abstracción y accesibles de síntesis y renderización de audio y video en el navegador.

Parte de los antecedentes están directamente relacionados con la experiencia performática de escribir de código al vuelo con fines creativos, audiovisuales y experimentales, de manera similar a como lo describen \cite{villasenor} para los casos de Barcelona y Ciudad de México. 

% De estas ideas se pueden extender reflexiones y puestas en práctica de ideas políticas relacionadas con la opotimización de recursos para el bajo consumo de energía y la resistencia a la obsolescencia programada de los dispositivos electrónicos. 

De manera directa, los siguientes proyectos están relacionados con la investigación y se convierten en casos de estudio, en algunas otras ocasiones fungen como antecedentes directos:

\begin{itemize}

\item Caso de estudio. THREE.studies \citep{threestudies} y en específico, las iteraciones threecln \citep{threecln} y threeBEASTs \citep{threeBEASTs}
\item Caso de estudio. Anti \citep{anti} 
\item Antecedente directo. Diálogo IV @ Coloquio Salvador Contreras. Artes Sonoras y Creación Musical en México: Siglo XXI \citep{dialogo}
\item Antecedente directo. Panorama. Escritura de espacios libres e inmersivos para el performance audiovisual \citep{PanoramaArticulo}\footnote{Una versión en construcción y en línea del artículo se puede encontrar en \url{https://piranhalab.github.io/panorama/}}
  
\end{itemize}

% De manera indirecta 

Algunos proyectos similares al que describe \textit{Tres Estudios Abiertos} son: 

\begin{itemize}

\item \textbf{Nivel Bajo:} Ruffbox \citep{ruffbox}, WebAssembly/Rust Tutorial\footnote{\url{https://www.toptal.com/webassembly/webassembly-rust-tutorial-web-audio} (Consultado el \today)} y Flocking \citep{flocking}.
\item \textbf{Nivel Medio:} \citep{supercolliderweb}, Web Audio API\footnote{\url{https://developer.mozilla.org/es/docs/Web/API/Web_Audio_API}}, Tone.js\footnote{\url{https://tonejs.github.io/}} y supercolliderjs\footnote{\url{https://github.com/crucialfelix/supercolliderjs/}}.
\item \textbf{Nivel Alto:} Estuary \citep{estuary}, Troop \citep{Troop}, flok \citep{flok}, tilt \citep{tilt}, LiveLab \citep{livelab}, Hydra \citep{hydra}, timeNot \citep{timenot} y seis8s \citep{seis8s}
  
\end{itemize}

