
\section*{Introducción}

\textit{Tres Estudios Abiertos} es un proyecto de investigación que busca adentrarse en nuevas prácticas experimentales y audiovisuales para el navegador. El proyecto busca demostrar que las lógicas de los lenguajes de programación posibilitan formas de pensamiento específicos con consecuencias estéticas que pueden extenderse y aportar a la investigación artística que involucra tecnología. Parte de la programación de un esqueleto de granulación audiovisual complementado con módulos personalizados de software que se implementarán en estudios para el navegador. 

El proyecto se centra en piezas artísticas para el navegador con características específicas: sin instalaciones, sin referencias a dependencias de terceros, basadas en la web, distribuidas y optimizadas para el bajo consumo de recursos de una computadora o una red de computadoras. Considera a \textit{Javascript}\footnote{¿Todavía es posible usar otros lenguajes de programación?} como el lenguaje de programación y busca reflexionar en el nivel bajo, medio y alto de programación. Como parte de la exploración de alto nivel, busca reflexionar sobre las ideas estéticas de los lenguajes de programación para la escritura de interfaces textuales de control. 

Consideramos que hay tres hilos que corren paralelamente en el bucle de ivnestigación-creación: estético, tecnológico e investigativo. En este último encontramos la relación de la escritura del texto resultante con la escritura de software. El vínculo se establece con el uso de sistemas distribuidos de control de versiones para la escritura de software y de texto. El documento resultante busca desbordarse del papel y de la palabra escrita por medio de una versión web. 
