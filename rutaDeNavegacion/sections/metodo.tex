\section*{Metodología}

%¿Cómo responder a la hipótesis principal? ¿Es posible medir de alguna manera la experiencia estética y subjetiva de un lenguaje de programación? ¿Es posible encontrar una respuesta en el alto nivel? 

La presente investigación hace referencia a la investigación artística y en específico retoma la idea de \textit{loop} o bucle para hablar de la relación entre investigación-creación. Esta forma de trabajo retroalimenta la escritura investigativa con la escritura investigativa y la práctica artística y de regreso. Consideramos que esta forma de proceder en la investigación revela aspectos que el distanciamiento convencional en la investigación no toma en cuenta.

En esta investigación hacemos referencia a procesos artísticos que involucran música, software y computadoras. En este sentido, las peculiaridades de la investigación se toman en cuenta para plantear un proyecto con escritura de software. El objetivo primordial del proyecto no busca realizar mediciones para establecer parámetros de ligereza o eficacia el software resultante. En todo caso busca aprovechar la lógica tecnológica de la programación y del procesamiento de información para encontrar soluciones para la investigación y la reflexión.

Uno de los aspectos que más rescatamos en este sentido es el uso del repositorio como una estrategia para trabajar con la escritura de software, un respaldo para el trabajo colaborativo y como una documentación que puede modificarse y consultarse en el tiempo. Para la presente investigación la noción de repositorio de código es central ya que nos permite adentarnos en \textit{las tripas del software} al mismo tiempo que atiende al código como un recurso de investigación. En este sentido la escritura y la consulta de software puede realizarse con sistemas distribuidos de control de versiones como \textit{Git}\footnote{``\textit{Git} es un sistema distribuido de control de versiones libre y abierto diseñado para tratar con todo, desde proyectos pequeños hasta proyectos muy grandes con velocidad y eficiencia.'' \url{https://git-scm.com/} (Consultado el \today)}. De esta manera buscamos lidiar con el caracter efímero del software y de piezas artísticas para el navegador. 

Aprovechamos la lógica de los sistemas distribuidos para el control de versiones para construir un entramado ``rizomático'' que pueda dar cuenta, por un lado, del proceso creativo con los respositorios de las piezas que existen en la web y por el otro, el texto que conforma la investigación y que pretende discutir con la contraparte de programación. De esta manera los procesos quedan lo suficientemente abiertos para interrelacionarse sin perder delimitación y diferenciación. En este sentido buscamos extender la propuesta de un momento anterior de investigación que anuncia la lógica del trabajo con sistemas distribuidos para la investigación artística con tecnología.

En este sentido la ejecución de la investigación coincide con los planteamientos y la delimitación del proceso de reflexión-creación: estos procesos implican múltiples objetivos y programas (Latour) que corren al mismo tiempo y que podemos enunciar de manera general como: tecnológico, estético y de investigación. Consideramos que la explicitación de esto 
