\section*{Metodología}

%¿Cómo responder a la hipótesis principal? ¿Es posible medir de alguna manera la experiencia estética y subjetiva de un lenguaje de programación? ¿Es posible encontrar una respuesta en el alto nivel? 

La presente investigación hace referencia a la investigación artística y en específico retoma la idea de \textit{loop} o bucle para hablar de la relación entre investigación-creación, tal y como se expresa en \citep{ocelotlMas}. Esta forma de trabajo retroalimenta la escritura investigativa con la práctica artística y de regreso. Consideramos que esta forma de proceder en la investigación revela aspectos que el distanciamiento convencional en la investigación no toma en cuenta.

En esta investigación hacemos referencia a procesos artísticos que involucran música, software y computadoras. En este sentido, las peculiaridades de la investigación se toman en cuenta para plantear un proyecto con escritura de software. El objetivo primordial del proyecto no busca realizar mediciones para establecer parámetros de ligereza o eficacia el software resultante. En todo caso busca aprovechar la lógica tecnológica de la programación y del procesamiento de información para encontrar soluciones para la investigación y la reflexión.

Uno de los aspectos que más rescatamos en este sentido es el uso del repositorio como una estrategia para trabajar con la escritura de software, un respaldo para el trabajo colaborativo y como una documentación que puede modificarse y consultarse en el tiempo. Para la presente investigación la noción de repositorio de código es central ya que nos permite adentarnos en \textit{las tripas del software} al mismo tiempo que atiende al código como un recurso de investigación. En este sentido la escritura y la consulta de software puede realizarse con sistemas distribuidos de control de versiones como \textit{Git}\footnote{``\textit{Git} es un sistema distribuido de control de versiones libre y abierto diseñado para tratar con todo, desde proyectos pequeños hasta proyectos muy grandes con velocidad y eficiencia.'' \url{https://git-scm.com/} (Consultado el \today)}. De esta manera buscamos lidiar con el caracter efímero del software y de piezas artísticas para el navegador. 

Aprovechamos la lógica de los sistemas distribuidos para el control de versiones para construir un entramado que pueda dar cuenta, por un lado, del proceso creativo con los respositorios de las piezas que existen en la web y por el otro, el texto que conforma la investigación y que pretende discutir con la contraparte de programación. De esta manera los procesos quedan lo suficientemente abiertos para interrelacionarse sin perder delimitación y diferenciación. En este sentido buscamos extender la propuesta de un momento anterior de investigación que anuncia la lógica del trabajo con sistemas distribuidos para la investigación artística con tecnología.

La ejecución de la investigación consiste en relacionar, código, texto y recursos multimedia. En un momento previo de decisión el proyecto ponderará la implementación de distintos sistemas de escritura de texto/código y elegiremos el que mejor se adapte a los alcances del proyecto. Hasta el momento la investigación considera tres sistemas que de alguna u otra manera se relacionan con la escritura de texto/código: \textit{LaTeX}\footnote{LaTeX es un sistema de composición tipográfica de alta calidad; incluye funcionalidades diseñadas para la producción de documentación técnica y científica. \url{https://www.latex-project.org/} (Consultado el \today)}, \textit{Git} y \textit{JupyterLab}\footnote{\textit{JupyterLab} es un entorno de desarrollo interactivo basado en la web para \textit{notebooks} de \textit{Jupyter}, código y datos. \url{https://jupyter.org/}. (Consultado el \today)}. La elección o combinación de entornos para la escritura de la tesis requerirán una ponderación que tome en cuenta la integración entre texto, código y acoplamiento con el lenguaje de programación principal del proyecto: \textit{Javascript}. De manera complementaria, la investigación toma en consideración los alcances de diseño, composición tipográfica y estilo personalizados. Como una alternativa al formato de presentación impreso (digital o físico), el proyecto busca que el proceso y el resultado de la investigación pueda ser compilado y consultado en línea como una página web.\footnote{Una prueba de esta propuesta se puede consultar en: \url{https://emilioocelotl.github.io/tres-estudios-abiertos/}}

La ejecución de la investigación coincide con los planteamientos y la delimitación del proceso de reflexión-creación: estos procesos implican múltiples objetivos que corren al mismo tiempo y que podemos enunciar de manera general como: tecnológico, estético y de investigación. El uso de conceptos que atraviesen estos rubros nos permitirá desplazarnos a partir de la retroalimentación que se genera entre tecnología y propuesta artística. El vínculo hacia lo reflexivo puede establecerse en las plataformas para escribir código como entornos de desarrollo integrados (IDEs) y sistemas de control de versiones, en la escritura por sí misma como tecnología, en el uso de conceptos y su vinculación con la metáfora. 

