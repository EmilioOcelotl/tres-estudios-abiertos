\section*{Planteamiento del problema}

De entre los proyectos similares destacamos aquellos que son de nivel alto para responder a la pregunta: ¿Cuál es la diferencia entre los proyectos mencionados y \textit{Tres Estudios Abiertos}?

Dos perspectivas podrían aclarar el punto de partida del proyecto. La primera es funcional y hace referencia a la solución de problemas partiendo de una comunidad que ejecuta, retroalimenta y enriquece al proyecto. La segunda, apuesta por la diversidad en el desarrollo de interfaces de control, para este caso destacamos las posibilidades de Estuary \citep{estuary}.

El presente proyecto busca responder, en un momento anterior a la realización de módulos de software, si hay diferencias estéticas heredadas de notaciones musicales y computacionales, lenguajes de programación,  estilos musicales, flujos de voltaje que desembocan en síntesis de audio / imagen (de manera similar a lo sintetizadores modulares) e incluso planteamientos críticos sobre decolonialidad.

Referimos a este conjunto de diferencias como decisiones de diseño en sintaxis de control que tienen consecuencias en la estética que resulta de controlar motores de audio y video. En este sentido, \textit{Tres Estudios Abiertos} es una búsqueda que orbita en estas desiciones y se adscribe a la diversidad en la escritura de y con software. 

De manera particular, eeste proyecto propone una solución para la síntesis granular audiovisual en el navegador. Aprovecha las posiblidades de procesamiento y conexión en red para generar módulos de software. Uno de los objetivos de la investigación consiste en la implementación del entramado de módulos en piezas audiovisuales para el navegador. Cada una de las piezas tendrán módulos adicionales que matizarán el esqueleto granular y que aportarán elementos tecnológicos para la diferenciación entre piezas. Del resultado entre la escritura de software y su implementación en piezas específicas es que el sistema se afinará y se probará. 

La ruta de navegación de este escrito implica tres niveles de exploración que coinciden con los niveles de los proyectos anteriormente citados como antecedentes a la investigación. Estos niveles no responden a una línea temporal.  

\begin{itemize}


\item \textbf{Nivel bajo.-} Como parte de la búsqueda por la ligereza y el bajo consumo de recursos por parte de la computadora, el proyecto busca explorar las posibilidades de la compilación de código fuente directamente en el navegador. El uso de \textit{Web Assembly} será el camino para esta fase del proyecto. En este nivel es posible proponer una librería granular compliada para el navegador. 
\item \textbf{Nivel medio.-} Implementación del esqueleto de módulos con \textit{frameworks} dedicados previamente escritos como Web Audio API\footnote{``La API de Audio Web provee un sistema poderoso y versatil para controlar audio en la Web, permitiendo a los desarrolladores escoger fuentes de audio, agregar efectos al audio, crear visualizaciones de audios, aplicar efectos espaciales (como paneo) y mucho más.'' \url{https://developer.mozilla.org/es/docs/Web/API/Web_Audio_API} (Consultado el \today)}, three.js\footnote{``El proyecto de three.js apunta a la creación de una librería 3D fácil de usar, ligera, multinavegador, multipropósito''. \url{https://threejs.org/} (Consultado el \today)} y Icecast\footnote{``\textit{Icecast} es un servidor para transmitir audio y video, actualmente soporta Ogg (Vorbis y Theora), Opus, WebM y MP3. Puede ser usado para crear una estación de radio por Internet o para correr de manera privada una rocola y muchas otras cosas.'' \url{https://icecast.org/} (Consultado el \today)}. Actualmente hay dos piezas (y repositorios) que corren en este nivel: \textit{THREE.studies} (threecln y threeBEASTs) y \textit{Anti} (en desarrollo). Este nivel es de utilidad para bosquejar el posible comportamiento del esqueleto granular con marcos robustos y estables. 
\item \textbf{Nivel alto.-} Implementación de funciones de control que puedan interactuar con el esqueleto de módulos para la granulación de audio y video. Idealmente éstas conformarán la sintaxis de una interfaz de texto que será deducida de \textit{Javascript}. La escritura de software y la referencia a otros proyectos de nivel alto permite tener una idea de cómo podría resolverse el control de librerías y de motores de audio y video. Este nivel permite explorar las posibilidades poéticas del texto como interfaz. 
  
\end{itemize}

Si bien el proyecto puede diferenciarse en estos tres niveles, el objetivo al que apunta es la interacción del software escrito en cada fase.

Para la realización de una librería de síntesis granular audiovisual, el proyecto parte de conceptos propuestos por \cite{microsound} y en específico aborda la noción de escala de tiempo para realizar acercamientos o alejamientos que impliquen obra artística, tecnología y que incluso alcancen a plantear un marco de observación para lo político y lo social. De manera resumida, esta propuesta se desplaza en un continuo que nos permite observar el sonido en una dimensión microscópica, lo político y lo social en una dimensión macroscópica y las piezas artísticas como un punto intermedio entre estas dos escalas. 

Los casos de estudio estarán alojados en la web. En este sentido, el proyecto pretende resolver el \textit{backend} del proyecto y en específico, busca explorar las posiblidades técnicas y conceptuales de la distribución el web a partir del concepto par-a-par\footnote{\textit{P2P} (par a par) ``La arquitectura de una red distribuida puede ser llamada Par a Par (P-to-P, P2P, ...)   si los participantes comparten una parte de los recursos de su propio software (poder de procesamiento, capacidad de almacenamiento, capacidad de conexión a la red, impresoras,...) Estos recursos compartidos son necesarios para proveer el Servicio y el contenido ofrecido por la red... Estos son accedidos por otros pares directamente sin pasar por entidades intermediarias." \citep{p2p}} La aplicación concreta para las piezas de arquitecturas distribuidas apuntan a la comunicación entre nodos para reducir carga, para no depender de la centralidad de un servidor y especialmente para compartir flujos de audio y video a través de la web. El proyecto en este sentido apunta hacia la realización de un sintetizador distribuído alojado en la web (Hugo) y del uso de la noción de nodo y red para la resolución tecnológica de las piezas pero también como una forma de observar la organización social en la web. 
