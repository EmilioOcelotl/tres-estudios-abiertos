\section*{Plantamiento del problema}

De entre los proyectos similares destacamos aquellos que son de Nivel Alto para responder a la pregunta: ¿Cuál es la diferencia entre los proyectos mencionados y \textit{Tres Estudios Abiertos}? Dos perspectivas podrían aclarar el punto de partida del proyecto. La primera es funcional y hace referencia a la solución de problemas partiendo de una comunidad que ejecuta, retroalimenta y enriquece al proyecto. La segunda, apuesta por la diversidad en el desarrollo de interfaces de control y que se manifiesta de una manera muy específica en el proyecto Estuary. El presente proyecto busca responder en un momento anterior a la realización de módulos de software si hay diferencias estéticas heredadas de notaciones musicales y computacionales, lenguajes de programación,  estilos musicales, flujos de voltaje que desembocan en síntesis de audio / imagen e incluso planteamientos críticos sobre decolonialidad. Referimos a este conjunto de diferencias como decisiones de diseño en sintaxis de control que tienen consecuencias en la estética que resulta de controlar motores de audio y video. En este sentido, \textit{Tres Estudios Abiertos} es una búsqueda que orbita en estas desiciones y se adscribe a la diversidad en la escritura de y con software. 
